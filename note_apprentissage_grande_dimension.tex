\documentclass{article}
\title{Apprentissage en grande domension}
\begin{document}
\maketitle
\pagebreak

\begin{equation}
\min_{\beta\in\mathbb{R}} f(\beta)
\end{equation}
Conditions:
$f$ convexe:
\begin{equation}
f(y)>=f(x)+\nabla f(x)^T(y-x)
\end{equation}


Definition 1:
\begin{equation}
\forall \theta\in [0,1]
\end{equation}


Def 3 $M$ 

Def 4 Lipschizsienne
\begin{equation}
\forall x,y ||f(x)-f(y)||_2<=L||x-y||_2
\end{equation}

Def 5 contractant
\begin{equation}
L Lipschitz avec 0<=L<1
\end{equation}

Them 1 Thm point fixe:
$f$ est $\alpha -$contractant,
\begin{equation}
\exists x^* tel que f^*=f(x^*)
\end{equation}
La suite definie par $x_{n+1}=f(x_n)$ converge vers $x^*$ et v\'erifie 
\begin{equation}
||x_n-x^*||_2<=\frac{\alpha^n}{1-\alpha}||x_0-x_1||_2
\end{equation}

Gradient Algo

Prop 5 Gradient monotone
$f$ diff est convexe, si et seulement si
\begin{equation}
\begin{split}
&(\nabla f(x)-\nabla f(y))^T(x-y)>=0 \\
&=\nabla f(x) f-\text{consistante}
\end{split}
\end{equation}

PREUVE
1.$\Rightarrow$:
\begin{equation}
f(y)>=f(x)+\nabla f(x)^T(y-x)
\end{equation}
\begin{equation}
f(x)>=f(y)+\nabla f(y)^T(x-y)
\end{equation}
\begin{equation}
-f(x)-f(y)<-f(x)-f(y)+\nabla f(x)^T(x-y)-\nabla f(y)^T (x-y)
\end{equation}
\begin{equation}
(\nabla f(x)-\nabla f(y))^T (x-y)>=0
\end{equation}

2. $\Leftarrow$:
On introduit une fonction $\Phi$:
\begin{equation}
\Phi(t)=f(x+t(y-x))
\end{equation}
\begin{equation}
\Phi'(t)=\nabla f(x+t(y-x))^T(y-x)
\end{equation}

Comme $\nabla f$ est monotone
\begin{equation}
\Phi'(t)>=\Phi'(0), t>=0
\end{equation}
\begin{equation}
f(y)-\Phi(1)=\Phi(0)+\int_0^1\Phi'(t)dt
\end{equation}
\begin{equation}
f(y)>=\Phi(0)+\Phi'(0)=f(x)+\nabla f(x)^T(y-x)
\end{equation}

Theor\`eme Bo\^ite quadratique sup\'erieure
\begin{equation}
f\sim L^1,\nabla f est L-lipschitz
\end{equation}
Alors
\begin{equation}
g(x)=\frac{L}{2}x^Tx-f(x) est convexe
\end{equation}
\begin{equation}
f(y)<=\nabla <=\nabla f(x)^T(y-x)+\frac{L}{2}||x-y||_2^2
\end{equation}

1. $\nabla f$ Lipschitz
\begin{equation}
||\nabla f(y)-\nabla f(x)||_2 <= L||y-x||_2
\end{equation}

2. 
\begin{equation}
\begin{split}
(\nabla f(y)-\nabla f(x))^T(y-x)&<=||\nabla f(y)-\nabla f(x)||_2||y-x||_2\\
&<=L||y-x||_2^2
\end{split}
\end{equation}
\begin{equation}
\nabla g(x)=Lx-\nabla f
\end{equation}
\begin{equation}
\begin{split}
&(\nabla g(x)-\nabla g(y))^T(x-y)\\
=&(Lx-\nabla f(x)-Ly+\nabla f(y))^T(x-y)\\
=&-(\nabla f(y)-\nabla f(x))^T(y-x)+L||x-y||_2^2\\
>=&0
\end{split}
\end{equation}

\begin{equation}
y=x-t\nabla f(x)
\end{equation}
\begin{equation}
f(x-t\nabla f(x))<=f(x)+t(1-\frac{Lt}{2})||\nabla f(x)||_2^2
\end{equation}
choix de $t$ tel que $0<=t<\frac{1}{2}$
\begin{equation}
x^+=x-t\nabla f(x)
\end{equation}
\begin{equation}
\begin{split}
f(x^+)<=&f(x)+f(1-\frac{Lt}{2})||\nabla f(x)||_2^2\\
<=& f(x)-\frac{t}{2}||\nabla f(x)||_2^2\\
<=f^*+\nabla f(x)^T (x-x^*)-\frac{t}{2}||\nabla f(x)||^2\\
=& f^*+\frac{1}{2t}(||x-x^*||_2^2-||x-x^*-t\nabla f(x)||_2^2)\\
=& f^* +\frac{1}{2t}(||x-x^*||^2_2-||x^+-x^*||_2^2)
\end{split}
\end{equation}

\begin{equation}
\begin{split}
\sum_{k=1}^N (f(x_k)-k^*)<=&\frac{1}{2t}\sum_{k=1}^N(||x_{k-1}-x^*||_2^2-||x_k-x^*||_2^2)\\
=&\frac{1}{2t}(||x_0-x^*||_2^2-||x_N-x^*||_2^2)\\
<=\frac{1}{2t}||x_0-x^*||_2^2
\end{split}
\end{equation}

\end{document}