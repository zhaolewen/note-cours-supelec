\begin{document}
\section{Methodologie du test statistique}

\begin{equation}
X_1,\ldots,\X_{n_1} ~\mathcal{N}(\mu_1,\sigma^2)
\end{equation}
\begin{equation}
Y_1,\ldots,Y_{n_2} ~\mathcal{N}(\mu_2,\sigma^2)
\end{equation}

Test d'hypoth\`ese
\begin{equation}
H_0 \{\mu_1=\mu_2\} vs H_1 \{\mu_1\neq\mu_2\}
\end{equation}

Pour rejeter (ou non) l'hypoth\`ese nulle, on utilise une statistique de test dont on connait la distribution sous $H_0$ (m\^eme asymptotiquement).
Ici, on utilisera le test de student de statistique associ\'ee
\begin{equation}
T=\frac{\bar{X}-\bar{Y}}{s\sqrt{\frac{1}{n_1}-\frac{1}{n_2}}}
\end{equation}
avec
\begin{equation}
s=\sqrt{\frac{(n_1-1)s_1^2+(n_2-1)s_2^2}{n_1+n_2-2}}
\end{equation}
\end{document}