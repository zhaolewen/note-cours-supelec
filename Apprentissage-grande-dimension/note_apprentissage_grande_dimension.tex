\documentclass{article}
\usepackage{amsfonts}
\usepackage{amsmath}
\usepackage{bbm}

\title{Apprentissage en grande domension}
\begin{document}
\maketitle
\pagebreak

\begin{equation}
\min_{\beta\in\mathbb{R}} f(\beta)
\end{equation}
Conditions:
$f$ convexe:
\begin{equation}
f(y)>=f(x)+\nabla f(x)^T(y-x)
\end{equation}


Definition 1:
\begin{equation}
\forall \theta\in [0,1]
\end{equation}


Def 3 $M$ 

Def 4 Lipschizsienne
\begin{equation}
\forall x,y ||f(x)-f(y)||_2<=L||x-y||_2
\end{equation}

Def 5 contractant
\begin{equation}
L Lipschitz avec 0<=L<1
\end{equation}

Them 1 Thm point fixe:
$f$ est $\alpha -$contractant,
\begin{equation}
\exists x^* tel que f^*=f(x^*)
\end{equation}
La suite definie par $x_{n+1}=f(x_n)$ converge vers $x^*$ et v\'erifie 
\begin{equation}
||x_n-x^*||_2<=\frac{\alpha^n}{1-\alpha}||x_0-x_1||_2
\end{equation}

Gradient Algo

Prop 5 Gradient monotone
$f$ diff est convexe, si et seulement si
\begin{equation}
\begin{split}
&(\nabla f(x)-\nabla f(y))^T(x-y)>=0 \\
&=\nabla f(x) f-\text{consistante}
\end{split}
\end{equation}

PREUVE
1.$\Rightarrow$:
\begin{equation}
f(y)>=f(x)+\nabla f(x)^T(y-x)
\end{equation}
\begin{equation}
f(x)>=f(y)+\nabla f(y)^T(x-y)
\end{equation}
\begin{equation}
-f(x)-f(y)<-f(x)-f(y)+\nabla f(x)^T(x-y)-\nabla f(y)^T (x-y)
\end{equation}
\begin{equation}
(\nabla f(x)-\nabla f(y))^T (x-y)>=0
\end{equation}

2. $\Leftarrow$:
On introduit une fonction $\Phi$:
\begin{equation}
\Phi(t)=f(x+t(y-x))
\end{equation}
\begin{equation}
\Phi'(t)=\nabla f(x+t(y-x))^T(y-x)
\end{equation}

Comme $\nabla f$ est monotone
\begin{equation}
\Phi'(t)>=\Phi'(0), t>=0
\end{equation}
\begin{equation}
f(y)-\Phi(1)=\Phi(0)+\int_0^1\Phi'(t)dt
\end{equation}
\begin{equation}
f(y)>=\Phi(0)+\Phi'(0)=f(x)+\nabla f(x)^T(y-x)
\end{equation}

Theor\`eme Bo\^ite quadratique sup\'erieure
\begin{equation}
f\sim L^1,\nabla f est L-lipschitz
\end{equation}
Alors
\begin{equation}
g(x)=\frac{L}{2}x^Tx-f(x) est convexe
\end{equation}
\begin{equation}
f(y)<=\nabla <=\nabla f(x)^T(y-x)+\frac{L}{2}||x-y||_2^2
\end{equation}

1. $\nabla f$ Lipschitz
\begin{equation}
||\nabla f(y)-\nabla f(x)||_2 <= L||y-x||_2
\end{equation}

2. 
\begin{equation}
\begin{split}
(\nabla f(y)-\nabla f(x))^T(y-x)&<=||\nabla f(y)-\nabla f(x)||_2||y-x||_2\\
&<=L||y-x||_2^2
\end{split}
\end{equation}
\begin{equation}
\nabla g(x)=Lx-\nabla f
\end{equation}
\begin{equation}
\begin{split}
&(\nabla g(x)-\nabla g(y))^T(x-y)\\
=&(Lx-\nabla f(x)-Ly+\nabla f(y))^T(x-y)\\
=&-(\nabla f(y)-\nabla f(x))^T(y-x)+L||x-y||_2^2\\
>=&0
\end{split}
\end{equation}

\begin{equation}
y=x-t\nabla f(x)
\end{equation}
\begin{equation}
f(x-t\nabla f(x))<=f(x)+t(1-\frac{Lt}{2})||\nabla f(x)||_2^2
\end{equation}
choix de $t$ tel que $0<=t<\frac{1}{2}$
\begin{equation}
x^+=x-t\nabla f(x)
\end{equation}
\begin{equation}
\begin{split}
f(x^+)<=&f(x)+f(1-\frac{Lt}{2})||\nabla f(x)||_2^2\\
<=& f(x)-\frac{t}{2}||\nabla f(x)||_2^2\\
<=f^*+\nabla f(x)^T (x-x^*)-\frac{t}{2}||\nabla f(x)||^2\\
=& f^*+\frac{1}{2t}(||x-x^*||_2^2-||x-x^*-t\nabla f(x)||_2^2)\\
=& f^* +\frac{1}{2t}(||x-x^*||^2_2-||x^+-x^*||_2^2)
\end{split}
\end{equation}

\begin{equation}
\begin{split}
\sum_{k=1}^N (f(x_k)-k^*)<=&\frac{1}{2t}\sum_{k=1}^N(||x_{k-1}-x^*||_2^2-||x_k-x^*||_2^2)\\
=&\frac{1}{2t}(||x_0-x^*||_2^2-||x_N-x^*||_2^2)\\
<=\frac{1}{2t}||x_0-x^*||_2^2
\end{split}
\end{equation}


Prop: Quand $f$ est differenciable
\begin{equation}
f(y)>=f(x)+\nabla f(x)^T(y-x)
\end{equation}

Definition: sous gradient
$g$ est un sous gradient de $f$ en $x$, ssi
\begin{equation}
\forall y, f(y)>=f(x)+g^T(y-x)
\end{equation}

Definition: sous differentielle
$f$ convexe, on definit la sous differentielle de $f$ en $x$ comme
\begin{equation}
\partial f(x)=\{g|\forall y, f(y)>=f(x)+g^T(y-x)\}
\end{equation}

Theoreme 3:
\begin{equation}
x^*=argmin f \Leftrightarrow 0 \in \partial f(x^*)
\end{equation}
Si $0 \in \partial f(x^*)$, alors
\begin{equation}
\forall y, f(y)>=f(x^*)+0^T(y-x^*) \Leftrightarrow x^=argmin f
\end{equation}

Prop 7: lin\'earit\'e non n\'egative
$f_1$ et $f_2$ convexes, $\alpha_1,\alpha_2>=0$
\begin{equation}
f>=\partial(\alpha_1 f_1+\alpha_2 f_2)(x)=\alpha_1 \partial f_1(x)+\alpha_2 \partial f_x(x)
\end{equation}
$+$ addition d'ensemble
\begin{equation}
E+F=\{e+f avec e\in E,f\in F\}
\end{equation}

Prop 8: combinaison affine:
Si $h(x)=f(Ax+b)$, alors
\begin{equation}
\partial h(x)=A^T\partial f(Ax+b)
\end{equation}

$f$ est une fonction $G$-Lipschitzienne

ALGO: M\'ethode du "sous-gradient"
\begin{equation}
x_k\leftarrow x_{k-1}-t_k g_{k-1}
\end{equation}
ou
\begin{equation}
g_{k-1}\in\partial f(x_k-1)
\end{equation}

Trois possibilit\'e pour $t_k$
\begin{enumerate}
\item $t_k=t$
\item "Longueur constante" $t_k||g_{k-1}||_2 est constante$
\item 
\begin{equation}
t_k\to_{k\to +\infty} 0
\end{equation}
\begin{equation}
\sum_{k=1}^{+\infty}=+\infty
\end{equation}
\begin{equation}
\sum_{k=1}^{+\infty} t_k^2=\text{limite finie}
\end{equation}
\end{enumerate}

Theoreme: f convexe et non differentielle
$f$ est G-Lipschitzienne $\Leftrightarrow$ $||g||_2<=G,\forall g\in\partial f(x)$

Preuve:
$\Leftarrow$

On suppose $\forall x, \forall g\in\partial f(x)$
\begin{equation}
||g||_2<=G
\end{equation}
Soit $x(g_x)$ et $y(g_y)$
\begin{equation}
g_x^T(x-y)>=f(x)-f(y)>=g_y^T(x-y)
\end{equation}
\begin{equation}
G||x-y||_2>=f(x)-f(y)>=-G||x-y||_2
\end{equation}
\begin{equation}
\forall x,y, ||f(x)-f(y)||<=G||x-y||_2
\end{equation}

$\Rightarrow$
$\exists g$ tel que $||g||_2>G$
\begin{equation}
y=x+\frac{g}{||g||_2}
\end{equation}
\begin{equation}
f(y)>=f(x)+g^T(y-x)=f(x)+||g||_2>f(x)+G
\end{equation}
Pas possible car $f$ est $G$-Lipschitzienne

Attention: 
La m\'ethode du sous-gradient n'est pas une m\'ethode de descente.

\begin{equation}
x^+=tg
\end{equation}
$g$ sous-gradient de $f$ en $x$.
\begin{equation}
\begin{split}
||x^+-x^*||^2_2=||x-tg-x^*||_2^2\\
=&||x-x^*||_2^2+t^2||g||_2^2-2tg^T(x-x^*)\\
<=&||x-x^*||_2^2+t^2||g||_2^2-2t(f(x)-f^*)
\end{split}
\end{equation}

Pour une iteration $k$:
\begin{equation}
2t_k (f(x_{k-1})-f^*)<||x_{k-1}-x^*||_2^2-||x_k-x^*||_2^2+t_k^2||g_{k-1}||_2^2
\end{equation}
en sommant les in\'egalit\'es
\begin{equation}
\begin{split}
2(\sum_{k=1}^N t_k)(f_{best}^(N)-f^*)<=& ||x_0-x^*||_2^2-||x_N-x^*||_2^2+\sum_{k=1}^N t_k^2||g_{k-1}||_2^2\\
<=& ||x_0-x^*||_2^2+\sum_{k=1}^N t_k^2||g_{k-1}||_2^2
\end{split}
\end{equation}

1. $t_k=t$
\begin{equation}
f_{best}^{(N)}-f^*<=\frac{||x_0-x^*||_2^2}{2Nt}+\frac{G^2t}{2}
\end{equation}

2. $t_k ||g_{k-1}||_2=s$
\begin{equation}
f_{best}^{(N)}-f^*<=\frac{G||x_0-x^*||_2^2}{2Ns}+\frac{Gs}{2}
\end{equation}

3. $t_k\to 0,\sum t_k\to +\infty, \sum t_k^2$ converge
\begin{equation}
f_{best}^{(N)}-f^*<=\frac{||x_0-x^*||_2^2+\sigma^2\sum t_k^2}{2\sum t_k}
\end{equation}

Conclusion: La m\'ethode du sous gradient n'est pas facile \`a param\'etrer pour obtenir sa convergence.

Exercise: 
\begin{equation}
f(\beta)=||X\beta-y||_2^2+\lambda||\beta||_1
\end{equation}
\begin{equation}
\partial f(\beta)=X^T(X\beta-y)+\lambda\partial_{||\dot||_1}(\beta)
\end{equation}
\begin{equation}
[\partial_{||\dot||_1}(\beta)]=
\left\{\begin{array}{lll}
sign(\beta_i) &\text{si} \beta_i\neq 0\\ \relax
[-1,1] & \text{si} \beta_i=0
\end{array}\right.
\end{equation}

Definition Operateur proximal
\begin{equation}
prox_f(x)=argmin_u \{f(u)+\frac{1}{2}||u-x||_2^2\}
\end{equation}
$f$ convexe "semi-continue inf\'erieurement"(sci). alors, $prox_f(x)$ existe et est unique.

Theoreme Caract\'erisation par le sous-gradient
\begin{equation}
u=prox_f(x) \Leftrightarrow x-u \in \partial f(u)
\end{equation}

Preuve:
\begin{equation}
\begin{split}
u=prox_f(x) &\Leftrightarrow u=argmin \{f(u)+\frac{1}{2}||u-x||_2^2\}\\
&\Leftrightarrow 0\in \partial g(u)\\
&\Leftrightarrow 0\in \partial g_1(u)+\partial g_2(u)\\
&\Leftrightarrow 0\in \partial f(u) + (u-x) \Leftrightarrow x-u \in\partial f(u)
\end{split}
\end{equation}
\begin{equation}
g(y)=g_1(y)+g_2(y)=f(y)+\frac{1}{2}||y-x||_2^2
\end{equation}

Algorithme du gradient proximal
\begin{equation}
0\in\partial f(x^*) \Leftrightarrow x^*=argmin_x f(x)
\end{equation}
\begin{equation}
\partial (f_1+f_2)=\partial f_1+\partial f_2
\end{equation}
Si $f$ est diff\'erentielle en $x$, alors
\begin{equation}
\partial f(x) =\nabla f(x)
\end{equation}

Norme euclidienne
\begin{equation}
f(x)=||x||_2
\end{equation}
\begin{equation}
prox_{tf}(x)=\left\{\begin{array}{rcl}
(1-\frac{t}{||x||_2})x&, ||x||_2>=t\\
0&, sinon
\end{array}\right.
\end{equation}

Multiplication par un scalaire >0
\begin{equation}
f(x)=\lambda g(x/\lambda)
\end{equation}
\begin{equation}
prox_f(x)=\lambda prox_{\frac{1}{\lambda}g}(\frac{x}{\lambda})
\end{equation}

Somme s\'eparable (Group LASSO)
\begin{equation}
f([x,y]=g(x)+h(y)
\end{equation}
\begin{equation}
prox_f([x,y])=[prox_g(x), prox_h(y)]
\end{equation}

Norme $l_1$
\begin{equation}
f(x)=||x||_1
\end{equation}
\begin{equation}
[prox_f(x)]_i\left\{\begin{array}{rcl}
x_i-1 & \text{si} x_i>=1\\
0 & \text{si} |x_i|<1\\
x_i+1 & \text{si} x_i<=-1
\end{array}\right.
\end{equation}

Num\'eriquement
\begin{equation}
prox l_1(x)=sign(x)\times pmax(abs(x)-1,x)
\end{equation}

\begin{equation}
min_\beta f(\beta)=min_\beta \{g(\beta)+h(\beta)\}
\end{equation}

Algorithme du gradient proximal
$g$ convexe et differentiable, $\nabla g$ est $L$-Lipschitzienne

$h$ convexe et non-differentiable (sci pour avoir $prox_{l_2}(x)$)

Exercise
\begin{equation}
f(\beta)=||X\beta-y||_2^2+\lambda ||\beta||_1
\end{equation}

Algorithme:
\begin{equation}
x_k\leftarrow prox_{t_kh}(x_{k-1-t_k \nabla g(x_{k-1})})
\end{equation}
\begin{equation}
f^*=f(x^*) \text{ fini}
\end{equation}
\begin{equation}
t_k=\frac{1}{L},(0<=t_k<\frac{1}{L})
\end{equation}

Gradient Map
\begin{equation}
G_t(x)=\frac{1}{t}(x-prox_{tl_2}(x-t\nabla g(x)))
\end{equation}

Pourquoi?
\begin{equation}
x^+=x-tG_t(x)
\end{equation}

Attention:
\begin{itemize}
\item $G_t(x)$ n'est pas un gradient pour $g$, n'est pas un sous-gradient pour $h$ ou pour $f$
\item $G_t(x^*)=0$ ssi $x^*=argmin f$
\end{itemize}

Borne Quadratique Sup\'erieure (BQS)
\begin{equation}
g(y)<=g(x)+\nabla g(x)^T (y-x)+\frac{L}{2}||y-x||_2^2
\end{equation}
Pour
\begin{equation}
y(=x^+)=x-tG_t(x)
\end{equation}
\begin{equation}
\begin{split}
g(x-tG_t(x))<=&g(x)-t\nabla g(x)^T G_t(x)+\frac{L}{2} t^2||G_t(x)||_2^2\\
<=g(x)-t\nabla g(x)^TG_t(x)+\frac{t}{2}||G_t(x)||_2^2
\end{split}
\end{equation}

Th\'eor\`eme: L'in\'egalit\'e pr\'ec\'edente nous permet de montrer
\begin{equation}
f(x-tG_t(x))<=f(z)+G_t(x)^T (x-z)-\frac{t}{2}||G_t(x)||_2^2
\end{equation}

\begin{equation}
\begin{split}
f(x-tG_t(x))<=&g(x)-t\nabla g(x)^TG_t(x)+\frac{t}{2}||G_t(x)||_2^2+h(x-tG_t(x))\\
<=& g(z)+\nabla g(z)^T(x-z)-t\nabla g(x)^T G_t(x)+\frac{t}{2}||G_t(x)||_2^2+h(z)+v^T(x-z-tG_t(x))\\
=& f(z)+G_t(x)^T(x-z)-\frac{t}{2}||G_t(x)||_2^2
\end{split}
\end{equation}

Pour
\begin{equation}
z=x
\end{equation}
on a
\begin{equation}
f(x^+)<=f(x)-\frac{t}{2}||G_t(x)||_2^2
\end{equation}
\begin{equation}
f(x^+)\to f(x_k)
\end{equation}
Donc, on a une m\'ethode de descente !

Pour $z=x^*$
\begin{equation}
\begin{split}
f(x^*)-f^*<=&G_t(x)^T(x-x^*)-\frac{t}{2}||G_t(x)||_2^2\\
=& \frac{1}{2t}(||x-x^*||_2^2-||x-x^*-tG_t(x)||_2^2)\\
=& \frac{1}{2t}(||x-x^*||_2^2-||x^+-x^*||_2^2)
\end{split}
\end{equation}

\begin{equation}
f(x_N)-f^*<=\frac{1}{2Nt}||x_0-x^*||_2^2
\end{equation}

\begin{equation}
[prox_{t||\dot||_1}](x)=\left\{\begin{array}{rcl}
x_i-t & \text{si} x_i>=t\\
0 & \text{si} |x_i|<t\\
x_i+t & \text{si} x_i <=t
\end{array}\right.
\end{equation}

Fast Proximal gradient algorithm

Convexe \& differentielle
\begin{equation}
f(y)>=f(x)+\nabla f(x)^T(y-x)
\end{equation}

Sous-gradient | sous differentielle
\begin{equation}
\partial f(x)=\{g|g^T(x-y)<=f(y)-f(x)\}
\end{equation}

Prox.
\begin{equation}
prox_f (x) = argmin_\mu \{f(\mu)+\frac{1}{2}||x-\mu||^2_2\}
\end{equation}
\begin{equation}
x-u\in\partial f(u) \Leftrightarrow u=prox_f (x)
\end{equation}

\begin{equation}
\min f(\beta)=g(\beta)+h(\beta)
\end{equation}
$\nabla g$ L-Lipschitzienne
$prox_{th}$ convexe

FISTA:
(n'est pas une m\'ethode de descente)
\begin{equation}
y = x_{k-1}+\frac{k-2}{k+1}(x_{k-1}-x_{k-2})
\end{equation}
\begin{equation}
x_k=prox_{t_k h}(y-t_k\nabla g(y))
\end{equation}
\begin{equation}
t_k=\frac{1}{L} \text{constant}
\end{equation}

Reformulation
\begin{equation}
\theta_k=\frac{2}{k+1}
\end{equation}
$v_k$ tel que $v_0=x_0$ et $\forall k>=1$
\begin{equation}
\left\{
\begin{array}{l}
y=(1-\theta_k)x_{k-1}+\theta_k v_{k-1}\\
x_k=prox_{th} (y-t_k\nabla g(y))\\
v_k=x_{k-1}+\frac{1}{\theta_k}(x_k-x_{k-1})
\end{array}\right.
\end{equation}

In\'egalit\'e
\begin{equation}
\forall k>=2,\frac{1-\theta_k}{\theta_k}<=\frac{1}{\theta^2_{k-1}}
\end{equation}
BQS(g)
\begin{equation}
g(u)<=g(z)+\nabla g^T(z)(u-z)+\frac{L}{2}||u-z||^2_2
\end{equation}
$BQS(h)$
\begin{equation}
u=prox_{th}(w)
\end{equation}
alors
\begin{equation}
\forall z, h(u)<=h(z)+\frac{1}{t}(v-u)^T(u-z)
\end{equation}

1.
\begin{equation}
g(x^+)<=g(y)+\nabla g^T(y)(x^+-y)+\frac{1}{2t}||x^+-y||^2_2
\end{equation}
2.
\begin{equation}
\begin{split}
h(x^+)<=h(z)+\frac{1}{t}(y-t\nabla g(y)x^+)^T(x^+-z)\\
=h(z)+\nabla g(y)^T(z-x^+)+\frac{1}{t}(x^+-y)^T(z-x^+)
\end{split}
\end{equation}
1+2:
\begin{equation}
\begin{split}
f(x^+)=g(x^+)+h(x^+)\\
<=g(y)+h(z)+\nabla g(y)^T (x^+-y+z-x^+)+\frac{1}{2t}||x^+-y||_2^2+\frac{1}{t}(x^+-y)^T(z-x^+)\\
<=f(z)+\frac{1}{2t}||x^+-y||_2^2+\frac{1}{t}(x^+-y)^T(z-x^+)
\end{split}
\end{equation}

\begin{equation}
\begin{split}
f(x^+)-f^*-(1-\theta)(f(x)-f^*)\\
<=\frac{\theta^2}{2t}(||v-x^*||_2^2)-||v^+-x^*||_2^2\\
\Leftrightarrow \frac{t}{\theta^2}(f(x)-f^*+\frac{1}{2}||v_1-x^*||_2^2<=\frac{1-\theta_1^2}{\theta_1^2}(f(z)-f^*)+\frac{1}{2}||v-x^*||_2^2
\end{split}
\end{equation}
Comme 
\begin{equation}
\frac{1-\theta_1}{\theta_1^2}<=\frac{1}{\theta_{i-1}^2}
\end{equation}
Conclusion
\begin{equation}
\frac{t}{\theta_k^2}(f(x_k)-f^*)-\frac{1}{2}||v_1-x^*||_2^2<=\frac{(1-\theta_1)^t}{\theta_1^2}(f(x_0)-f^*)+\frac{1}{2}||v_0-x^+||_2^2
\end{equation}
Ainsi
\begin{equation}
\frac{t}{\theta_k^2}f(x_k)-f^*<=\frac{(1-\theta_1)^t}{\theta_1^2}(f(x_0-f^*))+\frac{1}{2}||v_k-x^*||_2^2-\frac{1}{2}||v_0-x^*||_2^2
\end{equation}
\begin{equation}
f(x_k)-f^*<=\frac{2L}{(k+1)^2}||x_0-x^*||_2^2
\end{equation}


Travaux pratiques: Analyse en Composantes Principales parcimonieuse

1. Equation normale
\begin{equation}
\begin{split}
f(v)=&\frac{1}{2}||A-\delta v v^T||_F^2\\
=&\frac{1}{2}tr((A-\delta vv^T)(A-\delta v v^T))\\
=&tr(A^TA-\delta A^Tvv^T-\delta vv^TA+\delta^2vv^Tvv^T)\\
=&\frac{1}{2}tr(A^TA)-\frac{1}{2}tr(\delta A^Tvv^T)-\frac{1}{2}tr(\delta vv^TA)+\frac{1}{2}tr(\delta^2vv^Tvv^T)\\
=&\frac{1}{2}tr(A^TA)-\delta v^TAv+\frac{1}{2}\delta^2 (v^Tv)
\end{split}
\end{equation}

\begin{equation}
\delta=v^TAv/(v^Tv)^2
\end{equation}
\begin{equation}
Av=\frac{\alpha+\delta^2}{\delta}v
\end{equation}
Tel que $v^Tv=1, Av=\delta v$
$\delta$ est valeur propre de $A$ associ\'ee \`a $v$
\begin{equation}
f(v)=(v^Tv)^2
\end{equation}
\begin{equation}
\nabla f(v)=4(v^Tv)v
\end{equation}

\begin{equation}
\begin{split}
\nabla L(v)=&-2\delta Av+2\delta^2(v^Tv)v+2\alpha v\\
=&0 \Leftrightarrow \delta Av=(\delta^2(v^Tv)+\alpha)v
\end{split}
\end{equation}

\begin{equation}
f(v,\delta)=\frac{1}{2}||A-\delta vv^T||_F^2
\end{equation}
\begin{equation}
A=V\Delta V^T
\end{equation}
Avec 
\begin{equation}
\Delta=diag(\delta_1,\ldots,\delta_n)
\end{equation}

$\delta, v$ qui sont solution de $\min f$
\begin{equation}
A-\delta vv^T=Vdiag(0,\delta_2,\ldots,\delta_n)V^T=B
\end{equation}
\begin{equation}
tr(B^TB)=\sum_{k=2}^m\delta_k^2
\end{equation}
\begin{equation}
\delta_1=\delta_{\max}
\end{equation}

ACP
\begin{equation}
v^{k+1}=normalize(Av^{(k)})
\end{equation}

ACP $l_1$
\begin{equation}
V^{(k+1)}=normalize(prox_{\lambda||\dot||}(Av^{(k)}))
\end{equation}


\section{Les mod\`ele graphique gaussien}
Soit $X\sim\mathcal{N}(\mu,\Sigma)$, on suppose $\Sigma$ inversible, de dentit\'e
\begin{equation}
f_\alpha(x)=(2\Pi)^{-P/2}|\Sigma|^{-P/2}exp(-\frac{1}{2}(x-\mu)^T\Sigma^{-1}(x-\mu))
\end{equation}
Posons
\begin{equation}
K=\Sigma^{-1}
\end{equation}
La matrice de corr\'elation. On a
\begin{equation}
f_\alpha(x) \alpha |K|^{P/2}exp(-\frac{1}{2}(x-\mu)^T\Sigma(x-\mu))
\end{equation}
D\'efinition

Un mod\`ele graphique $G(V,E)$ o\`u $V={1,\ldots,P}$ l'ensemble de noeudes et $E=$ l'ensemble des liens connectant certaine paire de noeud. Le paire $(i,j)\in E$ $\Leftrightarrow$ $X_i,X_j$ sont "conditionnellement d\'ependants " sachant toutes les autres variables $X_{V\\ \{i,j\}}$
Autrement dit, $(i,j) \text{ not }\in E$ $\Leftrightarrow$ sont conditionnellement ind\'ependantes sachant $X_{V\\ \{i,j\}}$

Proposition
\begin{equation}
(i,j) \in E \Leftrightarrow K_{ij}\neq 0
\end{equation}
Preuve: Supposons $\mu=0$, ainsi
\begin{equation}
f_x(x)\alpha exp(-\frac{1}{2}x^TKx)
\end{equation}
La densit\'e conditionnelle de $(X_i,X_j)$ sachant toutes autres variables est d\'efinie par
\begin{equation}
f(x_i,x_j|x_1,\dots,x_p)\alpha exp(-\frac{1}{2}x_b^TK_{bb}x_b)
\end{equation}
Avec
\begin{equation}
K_{bb}=\begin{bmatrix}
K_{11} & K_{12}\\
K_{21} & K_{22}
\end{bmatrix}
\end{equation}

De dimension $2\times2$.
Ainsi
\begin{equation}
f(x_b|x_*)\alpha exp(-\frac{1}{2}x_1^TK_{11}x_1-\frac{1}{2}x_2^TK_{22}x_2-x_1^TK_{12}x_2)a=f(x_1)f(x_2)exp(-x_1^TK_{12}x_2)
\end{equation}

(Whatever)

Afin d'interpreter les \'elements de la matrice de ?, on \'etude maintenant la corr\'elation partielle.

Soit $p_{ij}$ la corr\'elation entre $X_i$ et $X_j$ apr\`es avoir \'elinmin\'e l'effet de toute les elements $\{X_k|k\inV\\ \{i,j\}\}$

\section{Analyse de donn\'ees structur\'ees}

\begin{equation}
X_1\to y_1=X_1w_1,\ldots,X_J\to y_J=X_Jw_J
\end{equation}
\begin{equation}
\max_{w_1,\ldots,w_J}\sum_{j=1}^J\sum_{k=1}^J cos(X_jw_j, X_kw_k)
\end{equation}

Rappel: Analyse en Composantes principales

Objectif: Trouver une combinaison lin\'eaire des colonnes de $X$ qui soit "repr\'esentative"

Crit\`ere: $w_1=argmax_w var(Xw)$ s.t. $||w||=1$

Solution: $w_1$ est premier vecteur propre de $\frac{1}{n}X^TX$ 
\begin{equation}
\frac{1}{n}X^TXw=\lambda w
\end{equation}

* Regression PLS-1

Objectif: Trouver une combinaison lin\'eaire des colonnes de $X$: $t=Xw$ bien explicative de son propre bloc et cor\'el\'e \`a $y$.

Crit\`ere:
\begin{equation}
w_1=argmax_w cov(Xw,y) \text{ s.t. } ||w||=1
\end{equation}
Solution:
\begin{equation}
w_1=\frac{X^Ty}{||X^Ty||}
\end{equation}

M\'ethode d'analyse de donn\'ees \`a 2 blocs
L'objectif des m\'ethodes d'analyse de donn\'ees structur\'ees en 2 blocs est de comprendre la relation. Il s'agit d'identifierdes sous-ensemble de variables dans chaque bloc qui "cr\'eent" le lien.

Une premi\`ere m\'ethode intitul\'ee R\'egression PLS 2 est d\'efinie par le crit\`ere suivant

$y_1=X_1w_1$: $y_1$ composante, $w_1$ vecteur de poids
$(w_1, w_2)=argmax_{w_1\in\mathbb{R}^{P_1},w_2\in\mathbb{R}^{P_2}} cov(X_1w_1,X_2w_2) \text{ s.t. } ||w_1||=||w_2||=2$

Pour r\'esoudre ce probl\`eme d'optimisation, on passe par la fonction Lagrangien donn\'ee par
\begin{equation}
L=\frac{1}{n}w_1^TX_1^TX_2w_2-\lambda_1(w_1^Tw_1-1)-\lambda_2(w_2^Tw_2-1)
\end{equation}
On va d\'eriver $L$ par rapport \`a $w_1$ et $w_2$
\begin{equation}
\frac{\partial L}{\partial w_1}=\frac{1}{n}X_1^TX_2w_2-2\lambda_1w_1
\end{equation}
\begin{equation}
\frac{\partial L}{\partial w_2}=\frac{1}{n}w_1^TX_1^TX_2-2\lambda_2w_2
\end{equation}
\begin{equation}
\left{\begin{array}{ll}
\frac{1}{n}X_1^TX_2w_2=2\lambda_1w_1\\
\frac{1}{n}X_2^TX_1w_1=2\lambda_2w_2
\end{array}\right.
\end{equation}

En multipliant de par et d'autre du signe \'egal par $w_1^T$ et $w_2^T$, on obtient
\begin{equation}
\left{\begin{array}{ll}
\frac{1}{n}w_1^TX_1^TX_2w_2=2\lambda_1\\
\frac{1}{n}w_2^TX_2^TX_2w_1=2\lambda_2
\end{array}\right.
\end{equation}
Ainsi
\begin{equation}
\lambda_1=\lambda_2
\end{equation}
En injectant une \'equation dans l'autre, on obtient que 
\begin{equation}
\frac{1}{4n^2}X_1^TX_2X_2^Tw_1=\lambda w_1
\end{equation}
\begin{equation}
\frac{1}{4n^2}X_2^TX_1X_1^TX_2w_2=\lambda w_2
\end{equation}

Conclusion:
$w_1$ est 1er vecteur propre de $X_1^TX_2X_2^TX_1$, $w_2$ est 1er vecteur propre de $X_2^TX_1X_1^TX_2$

Remarque: La r\'egression PLS 2 s'appuie sur un crit\`ere de covariance
\begin{equation}
cov^2(X_1a_1, X_2a_2)=var(X_1a_1)*cov^2(X_1a_1,X_2,a_2)*var(X_2a_2)
\end{equation}
On cherche une composante $y_1=X_1a_1$ bien explicative de son propre bloc $\to ACP(x_1)$

On cherche une composante $y_2=X_2a_2$ bien explicative de $X_2$. Ainsi $ACP(X_2)$

Plut\^ot que de maximiser la covariance entre composantes on peut voulouir maximiser la corr\'elation (Hoteling, 1936) propose le crit\`ere suivant
\begin{equation}
\begin{split}
(a_1, a_2)=argmax_{a_1,a_2} cor(X_1a_1,X_2a_2)=argmax\frac{cov(X_1a_1,X_2a_2)}{\sqrt{var(X_1a_1)}\sqrt{var(X_2a_2)}}\\
=argmax_{a_1,a_2} \frac{\frac{1}{n}a_1^TX_1^TX_2^Ta_2}{\sqrt{\frac{1}{n}a_1^TX_1^TX_1a_1}\sqrt{\frac{1}{n}a_2^TX_2^TX_2a_2}}
\end{split}
\end{equation}

On remarque que la solution est invariante par changement d'\'echelle et donc on peut consid\'erer le probl\`eme d'optimisation \'equivalent suivant
\begin{equation}
(a_1,a_2)=argmax_{a_1\in\mathbb{R}^{P_1},a_2\in\mathbb{R}^{P_2}} cov(X_1a_1,X_2a_2)
\end{equation}   
s.t.
\begin{equation}
\left{\begin{array}{ll}
var(X_1a_1)=1\\
var(X_2a_2)=1
\end{array}\right.
\end{equation}

Pour r\'esoudre ce probl\`eme d'optimisation, on passe comme pr\'ec\'edemment par le Lagrangien
\begin{equation}
L=\frac{1}{n}a_1^TX_1^TX_2a_2-\lambda_1(\frac{1}{n}a_1^Ta_1^TX_1^TX_1a_1-1)
\end{equation}

En annulant les d\'eriv\'ees Lagrangien par rapport \`a $a_1$ et $a_2$
\begin{equation}
\left{\begin{array}{ll}
\frac{1}{n}X_1^TX_2a_2=2\lambda_1\frac{1}{n}X_1^TX_1a_1\\
\frac{1}{n}X_2^TX_1a_1=2\lambda_2\frac{1}{n}X_2^TX_2a_2
\end{array}\right.
\end{equation}

En multipliant par $a_1^T$ et $a_2^T$, il vient que $\lambda_1=\lambda_2$

En injectant l'un dans l'autre
\begin{equation}
\frac{1}{4n^2}(X_1^TX_1)^{-1}X_1^TX_2(X_2^TX_2)^{-1}X_2^TX_1a_1=\lambda^2a_1=\frac{1}{4n^2}Q_{12}a_1
\end{equation}

En conclusion:
$a_1$ est $1er$ vecteur propre de $Q_{12}$, $a_2$ est 1er vecteur propre de $Q_{21}$.

R\'esum\'e g\'en\'erale: Jusqu'alors, on a vu 2 m\'ethodes 2 blocs bas\'es sur les unit\'es suivants
Crit\`ere 1:
\begin{equation}
cov(X_1a_1, X_2a_2) \text{ s.c. }||a_1||=||a_2||=1
\end{equation}

Finalement, l'analyse canonique (CCA) et PLS2 sont bas\'ees sur la m\^eme fonction objective mais avec des contraints diff\'erents.

Dans la suite nous allons pr\'esenter un cadre unifiant les $2$ m\'ethodes, Pour ce faire, introduisons des param\`etres $\tau_j\in[0,1],j=1,2$ et consid\'erons le probl\`eme d'optimisation suivant:
\begin{equation}
(a_1,a_2)=argmax_{a_1\in\mathbb{R}^{P_1},a_2\in\mathbb{R}^{P2}} cov(X_1a_1,X_2a_2)
\end{equation}
s.c.
\begin{equation}
\left{\begin{array}{ll}
(1-\tau_1)var(X_1w_1)+\tau_1||a_1||_2^2=1\\
(1-\tau_2)var(X_2w_2)+\tau_2||a_2||_2^2=1
\end{array}\right.
\end{equation}
Ce probl\`eme d'optimisation d\'efinit l'analyse canonique r\'egularis\'ee.

Si $\tau_1=\tau_2=0$, alors on retrouve l'analyse canonique.
Si $\tau_1=\tau_2=1$, alors pm retrrouve PLS2.
Si $\tau_1, \tau_2=0$, alors le crit\`ere sous-jacent devient
\begin{equation}
\max_{a_1} var(X_1a_1)cov^2(X_1a_1, X_2a_2)
\end{equation}
s.c.
\begin{equation}
||a_1||=1, var(X_2a_2)=1
\end{equation}
L'analyse des redondance (Wollenberg, 1977) s'appuient sur ce dernier crit\`ere.

\begin{table}
\begin{tabular}{ccc}
 & $\tau_1=0$ & $\tau_1=1$ \\
 $\tau_2=0$ & Analyse canonique & Analyse de redondance de $X_1$ sur $X_2$ \\
 $\tau_2=1$ & Analyse des redondances de $X_2$ sur $X_1$ & PLS2
\end{tabular}
\end{table}

Regardons maintenant ce qu'il se oasse quand $\tau_1\in(0,1)$ et $\tau_2\in(0,1)$.

On peut montrer que $a_1$ et $a_2$ sont vecteurs propres des matrices suivantes

Reprenons le probl\`eme d'optimisation de CCA regularis\'ee
\begin{equation}
\max cov(X_1a_1, X_2a_2)
\end{equation}
s.c.
\begin{equation}
(1-\tau_j)var(X_ja_j)+\tau_j||a_j||^2_2=1, j=1,2\\
=(1-\tau_j)\frac{1}{n}a_j^TX_j^TX_ja_j+\tau_ja_j^Ta_j=1
=a_j^T[(1-\tau_j)\frac{1}{n}X_j^TX_j+\tau_j i_{P_j}]a_j=1
\end{equation}

La solution de ce probl\`eme d'oiptimisation est obtenu en recherchant les vecteurs propres de
(en utilisant les m\^eme recettes que pr\'ec\'edamment)
\begin{equation}
((1-\tau_1)\frac{1}{n}X_1^TX_1+\tau_1I_{P_1})^{-1}X_1^TX_2((1-\tau_2)\frac{1}{n}X_2^TX_2+\tau_2I_{P_2})^{-1}X_2^TX_1
\end{equation}
\begin{equation}
\hat{\Sigma}_{11}X_1^TX_2\hat{\Sigma}_{22}X_2^TX_1
\end{equation}
On voit appara\^itre des estimations r\'egularis\'ees des $\Sigma_{11}$ et $\Sigma_{22}$.

Par symm\'etrique, on a que $a_2$ est 1er vecteur propre de 
\begin{equation}
\hat{\Sigma}_{11}X_1^TX_2\hat{\Sigma}_{22}X_2^TX_1
\end{equation}

Ainsi,
\begin{equation}
X_2^T[(1-\tau_2)\frac{1}{n}X_2X_2^T+\tau_2I_n]^{-1} X_1X_1^T[(1-\tau_1)\frac{1}{n}X_1X_1^T+\tau_1In]^{-1}X_2a_2=\lambda a_2
\end{equation}

Remarque

On obtient deux formulations \'equivalentes pour obtenir $a_1$ et $a_2$. Une formulation primale qu'on utilisera quand $n>p_j$ et une formulation duale \`a utiliser si $n<p_j$.

En plus, en -pr\'e-multipliant \`a gauche par $X_2$, on obtient le probl\`eme au valeur propre/vecteur propre suivant
\begin{equation}
X_2^T[(1-\tau_2)\frac{1}{n}X_2X_2^T+\tau_2I_n]^{-1} X_1X_1^T[(1-\tau_1)\frac{1}{n}X_1X_1^T+\tau_1In]^{-1}X_2a_2=\lambda X_2a_2
\end{equation}
Et posons
\begin{equation}
K_j=X_jX_j^T
\end{equation}
on obtient alors,
\begin{equation}
K_2[(1-\tau_2)\frac{1}{n}K2+\tau_2 I_n]^{-1}K_1 [(1-\tau_1)\frac{1}{n}K_1+\tau_1I_n]^{-1}y_2=\lambda y_2
\end{equation}

On constate que pour calculer les composantes $y_1$ et $y_2$, il suffit de reconna\^itre uniquement les matrices de produits scalaires entre observations pour chaque bloc $X_1$ et $X_2$.

On \'etend de fait les m\'ethodes cit\'es pr\'ec\'edemment au contexte des noyaux.

PLS $\to$ kernel PLS $\leftarrow$ Rosipal, 2001
CCA $\to$ kernel CCA
RA $\to$ kernel Redundancy Analysis


Exemple illustratif
Si $y$ est uni-vari\'ee, lakernel PLS se r\'eduit \`a 

Dans ce cours, l'analyse de tableaux multiples est pr\'esent\'e au travers de l'analyse canonique g\'en\'eralis\'eer\'egularis\'ee.(RGCCA) propos\'ee en 2011(Tenenhaus & Tenenhaus, 2011).

RGCCA est d\'efini par le probl\`eme d'optimisation suivant:
\begin{equation}
max_{a_1, \ldots,a_S} \sum_{j=1}^J \sum_{k=1}^J c_{jk} g(cov(X_ja_j,X_ka_k))
\end{equation}
s.c.
\begin{equation}
(1-\tau_j) var(X_ja_j)+\tau_j||a_j||_2^2=1,j=1,\ldots,J
\end{equation}
o\`u
\begin{equation}
c_{jk}=\left\{\begin{array}{lll}
0 si X_j not \leftrightarrow \\
1 si X_j \leftrightarrow 
\end{array}\right.
\end{equation}
$g$: fonction convexe

On va maintenant \'etudier en d\'etail ce probl\`eme d'optimisation qu'on peut re-exprimer comme:
\begin{equation}
\max_{a_1,\ldots,\a_J} f(a_1,\ldots,a_J) = \sum_{j,k=1}^Ja_{jk}g(\frac{1}{n}a^T_jX^T_jX_ka_k)
\end{equation}
s.c.
\begin{equation}
(1-\tau_j)\frac{1}{n}a_j^TX_j^TX_ja_j+\tau_ja^T_ja_j=1,j=1,\ldots,J\\
=a_j^T[(1-\tau_j)\frac{1}{n}X^T_jX_j+\tau_jI_{p_j}]a_j=1\\
=a^T_jM_ja_j=1
\end{equation}

Posons $b_j=M^{1/2}_ja_j$ et $P_j=X_jM_j^{1/2}$

On obtient alors
\begin{equation}
\max_{b_1,\ldots,b_J}f(b_1,\ldots,b_J)=\sum_{j,k=1}^Jc_{jk}g(\frac{1}{n}b_j^TP_j^TP_kb_k)
\end{equation}
s.c.
\begin{equation}
b_j^Tb_j=1
\end{equation}

Pour r\'esoudre ce probl\`eme, on va utiliser deux ingr\'edients
1. Block relatation
2. Majorization par minorization (MM)

Analyse de donn\'ees multibloc

Ecrivons le Lagrangien assou\'e au probl\`eme d'optimisation de RGCCA
\begin{equation}
L=\sum_{j,k=1}^J c_{jk}g(\frac{1}{n}b_j^TP_j^TP_kb_k)-\sum_{j=1}^J\lambda_j(b^T_jb_j-1)
\end{equation}
Annulons la d\'eriv\'ee de $L$ par rapport \`a $b_j$
\begin{equation}
\frac{\partial L}{\partial b_j}\sum_{k=1}^J c_{jk} g'(\frac{1}{n}b_j^TP_j^TP_kb_k)\frac{1}{n}P_j^TP_kb_k-2\lambda_jb_j=0
\end{equation}
\begin{equation}
b_j=\frac{1}{2\lambda_j}\sum_{k=1}^Jc_{jk}g'(\frac{1}{n}b_j^TP_j^TP_ka_k)P^T_jP_kb_k
\end{equation}
Posons
\begin{equation}
\epsilon_j=\sum_{k=1}^Jc_{jk}g'(\frac{1}{n}b_j^TP_j^TP_kb_k)P_kb_k
\end{equation}
Ainsi,
\begin{equation}
b_j=\frac{P_j^T\epsilon_j}{||P_j^T\epsilon_j||}
\end{equation}

Or
\begin{equation}
b_j=M_j^{1/2}a_j
\end{equation}
et
\begin{equation}
P_j=X_jM_j^{-1/2}
\end{equation}
Ainsi,
\begin{equation}
a_j=\frac{M_j^{-1\2}M_j^{-1/2}X_j^T\epsilon_j}{\epsilon_j^TX_j^TM_j^1X_j\epsilon_j}=\frac{M_j^{-1}X_j^T\epsilon_j}{\epsilon_j\ldots}
\end{equation}
Quelques commentaires
\begin{equation}
a_j\alpha [(1-\tau_j)\frac{1}{n}X^T_jX_j+\tau_jI_{P_j}]^{-1}X_j^T\epsilon_j
\end{equation}
Si $\tau_j=0$, ainsi $a_j\sim (X_j^TX_j)^{-1}X_j^T\epsilon_j$
$\Leftrightarrow$ $a_j$ est obtenu par r\'egression multiple de $\epsilon_j$ sur $X_j$

Si $\tau_j=1$, alors la contrainte devient  $||a_j||=1$

\begin{equation}
a_j=\frac{X_j^T\epsilon_j}{||X_j^T\epsilon_j||}
\end{equation}

............

Reprenons la forme g\'en\'erale pour $a_j$
\begin{equation}
a_j=[\epsilon_j^TX_j^TM_j^1X_j\epsilon_j]^{-1/2}M_j^{-1}X_j^T\epsilon_j
\end{equation}
\begin{equation}
a_j=[\epsilon_j^TX_j^T[(1-\tau_j)\frac{1}{n}X_j^TX_j+\tau_2 I_{P_j}]^{-1}X_j^T\epsilon_j]^{-1/2}[(1-\tau_j)\frac{1}{n}X_j^TX_j+\tau_j I_{P_j}]^{-1}
\end{equation}


Sparse Partial Least Squares
\begin{equation}
\max cov(Xa,y)
\end{equation}
s.c.
\begin{equation}
||a||_2=1
\end{equation}
\begin{equation}
||a||_1<s
\end{equation}

P1. Montrer que la solution optimale de SPLS est donn\'ee par 
\begin{equation}
a^*=\frac{S(\frac{1}{n}X^Ty,\lambda_1)}{||S(\frac{1}{n}X^Ty,\lambda_1)||_2}
\end{equation}
o\`u $S$ est l'operation de seuillage doux.

Q2: Impl\'ementation cet algorithme ou utiliser le package RGCCA(SGCCA).

Q3: Tester votre algorithme sur le jeu de donn\'ees Alzhieimer

Q4: Par une proc\'edure de d\'eflqtion, construire une deuxi\`eme composante PLS.

Q5: Visualisation des individus sur le plan $(y_1,y_2)$

Q6: Comparer les r\'esultats \`a ceux obtenus par les packages.

\end{document}
