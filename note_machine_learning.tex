\documentclass{article}
\title{Machine Learning}
\maketitle
\pagebreak

\begin{document}
\section{La regression dans tous ses \'etats}
\subsection{Introduction \`a la r\'egression multiple}
La r\'egression muyltiple permet d'\'etudier la liaison entre une variable (\`a expliquer) $Y$ et un ensemble de $p$ variables explicatives $X_1,\ldots,X_p$.

Le mod\`ele de la r\'egression multiple est d\'efinie par
\begin{equation}
Y=\beta_0+\beta_1 X_1 +\cdots+\beta_p X_p+\epsilon
\end{equation}
o\`u les coefficients de r\'egression $\beta_j$ sont des param\`etres fixe mais inconnus, et $\epsilon$ un terme al\'eatoire suivant une $\mathcal{N}(0,\sigma^2)$

\subsection{Donn\'ees et mod\`ele statistique}

On dispose de $n$ observations des variables $X_1,\ldots,X_p$
Soit 
\begin{equation}
X=
\begin{split}
y_1 & x_{11} & \ldots & x_{p1} \\
\ldots & \ldots & \ldots & \ldots \\
y_m & x_{1m} & \ldots & x_{pm}
\end{split}
\end{equation}
(Tableau d'observations)

\end{document}