\documenttype{article}
\usepackage{amsfont}

\title{Bio-Statistique}
\maketitle

\begin{document}
\section{Methodologie du test statistique}

\begin{equation}
X_1,\ldots,\X_{n_1} ~\mathcal{N}(\mu_1,\sigma^2)
\end{equation}
\begin{equation}
Y_1,\ldots,Y_{n_2} ~\mathcal{N}(\mu_2,\sigma^2)
\end{equation}

Test d'hypoth\`ese
\begin{equation}
H_0 \{\mu_1=\mu_2\} vs H_1 \{\mu_1\neq\mu_2\}
\end{equation}

Pour rejeter (ou non) l'hypoth\`ese nulle, on utilise une statistique de test dont on connait la distribution sous $H_0$ (m\^eme asymptotiquement).
Ici, on utilisera le test de student de statistique associ\'ee
\begin{equation}
T=\frac{\bar{X}-\bar{Y}}{s\sqrt{\frac{1}{n_1}-\frac{1}{n_2}}}
\end{equation}
avec
\begin{equation}
s=\sqrt{\frac{(n_1-1)s_1^2+(n_2-1)s_2^2}{n_1+n_2-2}}
\end{equation}
Parce que la loi de $T$ est connue sous $H_0$, on peut controler l'erreur de 1er esp\`ece et on peut calculer des p-values.

\begin{center}
\begin{tabular}
(Realit\'e/D'ecision) & non rejet de $H_0$ & rejet de $H_0$\\
$H_0$ vraie & OK & erreur de 1er esp\`ece (faux positif)\\
$H_0$ faux & erreur de 2e esp\`ece (faux n\'egatif) & OK
\end{tabular}
\end{center}
Notons $\alpha=\mathbb{P}(rejet de H_0 | H_0 vraie)$ erreur de 1er esp\`ece.

\textbf{p-value}
Probabilit\'e sous $H_0$ d'avoir une value de $T$ plus grande que celle observ\'ee.

\begin{equation}
p=\mathbb{P}_{H_0}(|T|>|t_{obs}|)
\end{equation}

On comprend donc bien que plus $p$ est petit, plus on aura tendance \`a rejeter l'hypoth\`ese nulle. $=>$ Rejet de $H_0$ si $p<seuil$. La p-value est une normalisation particulier de la statistique de sorte que sous $H_0$, $p~\mathcal{U}[0,1]$
\begin{center}
\begin{tabular}
seuil & $\mathbb{P}_{H_0}(p<seuil)$ \\
1 & 1\\
0.5 & 0.5 \\
$\alpha$ & $\alpha$
\end{tabular}
\end{center}
Si on veut controler l'erreur de premi\`ere esp\`ece au seuil $\alpha$, rejet de $H_0$ si $p<=\alpha$

La probl\`ematique des tests multiples
Pour un test:
\begin{center}
\begin{tabular}
(R\'ealit\'e/D\'ecision) & $H_0$ non rejet\'ee & $H_0$ rejet\'ee\\
$H_0$ vraie & TN & FP\\
$H_0$ faux & FN & TP
\end{tabular}
\end{center}

Pour une famille de $m$ tests
\begin{center}
\begin{tabular}
(R\'ealit\'e/D\'ecision) & $H_0$ non rejet\'ee & $H_0$ rejet\'ee & \\
$H_0$ vraie & $U$ & $V$ & m_0 \\
$H_0$ fausse & $T$ & $S$ & m_1 \\
& $W$ & $S$ & $m$
\end{tabular}
\end{center}
$m_0$ nombre de variables (ex. g\`enes) v\'erifiant l'hypoth\'ese nulle.
$m_1$ nombre de variables v\'erifiant l'alternative.
$m_0,m_1$ inconnue.

$m$ nombre de g\`ene test\'es.
$R$ nombre d'hypoth\`ese nulle rejet\'ee.

On suppose que $m_0$ hypoth\`ese $H_0$ vraies.
On suppose que tous les tests sont men\'es au niveau $\alpha$.
\begin{equation}
\left\{\begin{array}[rcl]
E_1=1 & si p_1<\alpha\\
E_1=0 & sinon
\end{array}\right.
=> E_1 ~ \mathcal{B}(\alpha)
\end{equation}

\begin{equation}
\left\{\begin{array}[rcl]
E_2=1 & si p_2<\alpha\\
E_2=0 & sinon
\end{array}\right.
=> E_2 ~ \mathcal{B}(\alpha)
\end{equation}

...

\begin{equation}
\left\{\begin{array}[rcl]
E_{m_0}=1 & si p_{m_0}<\alpha\\
E_{m_0}=0 & sinon
\end{array}\right.
=> E_{m_0} ~ \mathcal{B}(\alpha)
\end{equation}
\begin{equation}
=>\V=-\sum_{k=1}^{m_0} E_k ~ \mathcal{B}_{in}(m_0,\alpha)
\end{equation}

On a $\mathbb{E}[V]=m_0\alpha$
Conclusion: le nombre attendu de faux positifs augmente lin\'eairement avec $m_0$. 

Dans ce contexte, on s'int\'eresse au FWER (Family Wise Error Rate) d\'efini par:
\begin{equation}
FWER = \mathbb{P}(V>0)
\end{equation}
= Probabilit\'e de rejeter au moins une fois l'hypoth\`ese nulle \`a tort dans la famille des tests r\'ealis\'es.

Procedure de contr\^ole du FWER: Bonferroni.

\end{document}