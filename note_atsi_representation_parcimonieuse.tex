\documentClass{article}
\usepackage{amsmath}
\title{ATSI}
\maketitle

\begin{document}
\begin{equation}
\min_\alpha \frac{1}{2}||y-\phi \alpha||^2+\lambda ||\alpha||
\end{equation}
\begin{equation}
\Phi^H\Phi = \Phi\Phi^H=I
\end{equation}
Class de probl\`eme $\min_x f(x)+\phi(x)$, avec 
-$f$ L-lipschitz differentiable, convexe
-$\phi$ convexe

Def $f$ est L-Lipschitz differentiable ssi
\begin{equation}
\exists L>0 tel que
\end{equation}
\begin{equation}
\forall x,y\in \mathbb{R}^N, ||\Delta f(x)-\Delta f(y)||<=L||x-y||
\end{equation}
Lemme Si $f$ est $L$-Lipschitz differentiable, alors
\begin{equation}
|f(x)-f(y)-<\Delta f(y),x-y>|<=\frac{L}{2}||x-y||^2
\end{equation}
si de plus $f$  est convexe, alors
\begin{equation}
0<=f(x)-f(y)-<\Delta f(y),x-y> <= \frac{L}{2}||x-y||^2
\end{equation}
\begin{equation}
f(x)+\phi(x)<=f(y)+<\Delta f(y),x-y>+\frac{L}{2}||x-y||^2+\phi(x)
\end{equation}
Soit
\begin{equation}
\begin{split}
x^{(t+1)}&=\argmin_x f(x^{(t)})+<\Delta f(x^{(t)}),x-x^{(t)}>+\frac{L}{2}||x-x^{(t)}||^2+\phi(x)\\
&\\argmin_x\frac{1}{L}||x-x^{(t)}+\Delta f(x^{(t)})||^2_2+\frac{1}{L}\phi(x)
\end{split}
\end{equation}
\begin{equation}
\begin{split}
f(x^{(t+1)})+\phi(x^{(x+1)})&<=f(x^{(t)})+<\Delta f(x^{(t)}),x^{(t+1)-x^{(t)}}> +\frac{L}{2}||x^{(t+1)}-x^{(t)}||+\phi(x^{(t+1)})\\
&<=f(x^{(t)})+\phi(x^{(t)})
\end{split}
\end{equation}

Def Soit $\phi: \mathbb{R}^N->\mathbb{R}$ une fonction convexe. On appelle operator de proximit\'e de $\phi$ not\'e $prox_\phi$, al fonction 
\begin{equation}
\begin{split}
\mathbb{R}^N->\mathbb{R}^N\\
y->prox_\phi(y)=\argmin_x \frac{1}{2} ||y-x||^2_2+\phi(x)
\end{split}
\end{equation}

\begin{equation}
x^{(t+1)}=prox_{\frac{1}{2}\phi}(x^{(t)}-\frac{1}{L}\Delta f(x^{(t)}))
\end{equation}

\begin{equation}
\phi(x)=\lambda ||x||_2^2
\end{equation}

\begin{equation}
prox_\phi(y)=\argmin_x \frac{1}{L}||y-x||^2+\frac{\lambda}{2}||x||^2
\end{equation}

\begin{equation}
\frac{\partial}{\partial x}=0=x-y+\lambda x => x=\frac{1}{1+\lambda}y
\end{equation}

\begin{equation}
\phi(x)=\lambda ||x||_1
\end{equation}

\begin{equation}
prox_\phi(y) =\argmin_x\frac{1}{2} ||y-x||^2_2 +\lambda ||x||_1 => x=y(1-\frac{\lambda}{|y|})^+
\end{equation}

\textbf{Def}. Soit $f$ une fonction convexe, le point $s$ est un sous-gradient de $f$ en $x$ ssi 
\begin{equation}
\forall y\in \mathbb{E}, f(y) >= f(x)+<s,y-x>
\end{equation}

\textbf{Def} (sous-differentiel) 
\begin{equation}
\partial_f (x) = \{s\}
\end{equation}







\end{document}