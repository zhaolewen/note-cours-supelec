\documentclass{article}
\linespread{1.5}

\usepackage{amsmath}
\usepackage{amsfonts}
\usepackage{mathrsfs}
\usepackage{dsfont}
\usepackage{hyperref}
\usepackage{color}
\usepackage{amsthm}

\theoremstyle{plain}
\newtheorem{thm}{Th\'eor\`eme}[section]

\theoremstyle{definition}
\newtheorem{defn}[thm]{D\'efinition}
\newtheorem{propos}[thm]{Proposition}
\newtheorem{remq}[thm]{Remarque}

\title{Mineur Finance 1}

\begin{document}
\maketitle

\tableofcontents
\pagebreak

\section{Cadre de valorisation des produite devive}
\subsection{Introduction}
Les produits d\'eriv\'es comme des options expose leur vendeur \`a un grand risque de perte, si ce risque n'est pas couvert, la perte peut \^etre consid\'erable. Ainsi, le vendeur de ces produits doit r\'epondre \`a deux questions majeurs:
\begin{enumerate}
	\item Quel prix doit je facturer (probl\`eme de valorisation ou pricing)
	\item Comment investir cette prime pour couvrir tout au long du temps le risque et pouvoir honorer l'engagement \`a maturit\'e(probl\`eme de couverture ou hedging).
\end{enumerate}

Les deux probl\`emes sont intim\'ement li\'es.
 
\subsection{Hypoth\`ese}
\begin{enumerate}
	\item Absence d'opportunit\'e d'arbitrage
	
	C'est une hypoth\`ese fondamentale sur laquelle se basent de nombreux riasonnements. Cette stipule qu'on ne peut pas gagner de l'argent \`a coup s\^ur sans prendre de risque.
	
	\item March\'e complet
	
	Il s'agit de march\'e o\`u on arrive \`a mettre en place une strat\'egie pour atteindre un flux futur \`a une \'eche\'eance(Dans le cas du call, le flux futur est $(S_T-K)_+$)
\end{enumerate}
\subsection{Propri\'et\'e}
\begin{thm}
	Parit\'e Call-Put
	\begin{equation}
		C(t) - P(t) = S(t)-KB(t,T)
	\end{equation}
\end{thm}
o\`u
\begin{itemize}
	\item $C(t)$ prix du call
	\item $P(t)$ prix du put
	\item $S(t)$ prix de l'actif
	\item $K$ prix strike
	\item $B(t,T)=e^{-r(T-t)}$ valeur of a zero-coupon bond that matures to $1$ at  $T$
\end{itemize}
{\color{blue} \href{https://en.wikipedia.org/wiki/Put%E2%80%93call_parity}{Put-Call Parity on Wikipedia}}

\textbf{Explication}:

Par principe de AOA, $2$ portefeuilles de la m\^eme payoff au strike ont la m\^eme valeur \`a tout instant.

Donc on constrire 2 portefeuilles:
\begin{enumerate}
	\item Acheter une option Call sur l'actif $S$ et vendre une option Put avec strike $K$.
	\item Acheter un actif $S$ et vendre K bonds $B(t,T)=e^{-r(T-t)}$
\end{enumerate}
Ces deux portefeuilles donne la m\^eme valeur au strike. 
	
\subsection{Options am\'ericaines}
Option am\'ericaine similaire aux options europeens mais l'exercise du droit pourra se faire \`a toute date $t$ avant maturit\'e $T$.

On note $CallAmer_t(T, K)$ et $PutAmer_t(T, K)$ les prix des calls et puts am\'ericains au $t$.

On a 
\begin{equation}
CallAmer_t(T,K) \geq Call_t(T, K)
\end{equation}
 
\begin{equation}
PutAmer_t(T, K) \geq Put_t(T, K)
\end{equation}

Propri\'et\'e:
En l'absence de divident, on a:

\begin{equation}
CallAmer_t(T, K) = Call_t(T, K)
\end{equation}

\textbf{Meilleure explication sur Stackoverflow}

{\color{blue}\href{http://money.stackexchange.com/questions/5161/why-are-american-style-options-worth-more-than-european-style-options}{Stackoverflow question link}}

\textbf{Preuve}:

\begin{equation}
Call_t(T, K) \geq (S_t- K)_+
\end{equation}

Il n'est pas optimal d'exercer avant la maturit\'e. Donc $CallAmer_t(T, K) = Call_t(T, K)$

Remarque: Pour le Put, la propri\'et\'e n'est pas vraie.

\begin{equation}
Put_t(T, K) \leq KB(t, T)
\end{equation}

Pour un certain scenario, le prix de put est inf、‘erieur au prix d'exercise. $Put_t(T, K)<(S_t-K)_+$.

Pour $S_t$ tel que $KB_t(T)< K-S_t \Leftrightarrow S_t < (1-B_t(T))K$
Donc, on a
\begin{equation}
Put_t(T, K)< K-S_t
\end{equation} 

\section{Th\'eorie de la valorisation dans le cas discr\`et \`a une p\'eriode}

\subsection{Introduction}
Nous avons vu que les contrats \`a termes peuvent \^{e}tre valoris\'es par le principe de non arbitrage alors que ce principe ne donne que des bornes sur le prix des options. La valorisation de ces produits assez complexes n\'ecessite l'introduction des mod\`eles probabilistes pour d\'ecrire les sc\'enarios possibles du march\'e.

Dans ce march\'e, si l'on trouve une strat\'egie qui r\'eplique le payoff de l'option, alors la portefeuille construite par cette strat\'egie r\'eplique l'option.

Par l'absence d'opportunite d'arbitrage, la valeur de ce portefeuille est la m\^{e}me que celle de l'option. Cependant, ce prix est d\'ependant du mod\`ele qui doit \^etre proche de la r\'ealit\'e.

\subsection{Mod\`ele \`a une p\'eriode}

On consir\`ere qu'il y a $d+1$ actifs financiers et $2$ dates,: $t=0, t=1$. L'al\'ea du mod\`ele est represent\'e par $K$ \'etats dans $\Omega=\{\omega_1,\ldots,\omega_K\}$.

En $t=1$, avec des probabilit\'es $ \{p_1,\ldots,p_K\},(p_i>0) $.

Le prix de l'actif $i$ au $t$ est not\'e $S_t^i, 0\leq i\leq d$. $S^0$ represente l'actif sans risque $(S_0^0=1, S_1^0=e^R\approx 1+R)$
Une strategie $\theta\in\mathbb{R}^{d+1}$ est un vecteur contenant le nombre d'unit\'e de chaque actif dans ce portefeuille $V$ entre $t=0$ et $t=1$. 

La valeur de $V$ en $t$
\begin{equation}
V_t=V_t^\theta=\sum_{i=0}^d\theta_i S_t^i
\end{equation}
Le gain est defini par 
\begin{equation}
G=V_1-V_0=\sum_{i=0}^d \theta_i (S_1^i-S_0^i)=\sum_{i=0}^d \theta_i \Delta S^i
\end{equation}

On note
\begin{equation}
	\tilde{S}_t^i=\frac{S_t^i}{S_t^0},\tilde{V}_t=\frac{V_t}{S_t^0},\tilde{G}_t=\frac{G}{S_t^0}
\end{equation}
alors
\begin{equation}
\left \{ \begin{array}{ll}
&\tilde{V}_t =\theta_0+\sum_{i=1}^{d}\theta_i \tilde{S}_t^i\\
&\tilde{G} =\tilde{V}_1-\tilde{V}_0=\sum_{i=1}^d \theta_i (\tilde{S}_1^i-\tilde{S}_0^i)=\sum_{i=1}^d\theta_i\Delta \tilde{S}^i
\end{array}\right.
\end{equation}

\subsection{R\'esultats de non-arbitrage}
\subsubsection{D\'efinition}

Une opportunite d'arbitrage est une strategie $\theta$ tel que 
\begin{enumerate}
	\item  $V_0^\theta =0$
	\item $P(V_1^\theta >=0) = 1$ et $P(V_1^\theta>0)>0$ 
\end{enumerate}
o\`u
\begin{enumerate}
	\item $P(G\geq 0) = 1$
	\item $P(G>0) > 0$
\end{enumerate}

Commencer avec un \'etat avec valeur $0$, et la probabilite de gagner de l'argent est superierue \`a $0$.

\subsubsection{D\'efinition 2}

Une mesure de probabilit\'e $Q$ est dite risque neutre si:
\begin{enumerate}
	\item $Q\sim P$, c'est a dire, $Q(\omega_i)>0, \forall 1\leq i\leq K$
	
	$Q$ r\'eplique $P$
	
	\item $E^Q[\tilde{S}_1^i]=\tilde{S}_0^i, \forall i$
	
	Valeur relative des actifs ne change pas.
\end{enumerate}

\subsubsection{Th\'eor\`eme}
Le march\'e n'admet pas d'opportunit\'e d'arbitrage si et seulement s'il existe au moins une mesure de probabilit\'e risque neutre.

Preuve:

($\Rightarrow$) Supposons qu'il existe une mesure de probabilit\'e risque neutre. Supposons qu'il existe une opportunit\'e d'arbitrage, donc 
\begin{equation}
\label{eq:gain_risque_neutre}
Q(G^\theta\geq0)=1, Q(G^\theta>0)>0
\end{equation}
Mais
\begin{equation}
E^Q[\tilde{G}^\theta]=E^Q[\tilde{S}_1^\theta]-\tilde{S}_0^\theta=0
\end{equation}
ce qui est en contradiction avec (\ref{eq:gain_risque_neutre}).

($\Leftarrow$) Supposons qu'il y a absence d'opportunit\'e d'arbitrage. Soient
\begin{equation}
\mathcal{E}=\{(q_1,\ldots,q_K)|\sum_{i=1}^K q_i, q_i>0\}
\end{equation}
\begin{equation}
C=\{E^Q[\Delta\tilde{S}]|Q\in\mathcal{E}\}, \Delta \tilde{S}=(\Delta \tilde{S}^1,\ldots, \Delta\tilde{S}^d)^T
\end{equation}
Il faut d\'emontrer que $0\in C$. Dans ce cas l\`a, il existe une mesure de probabilit\'e de sorte que la valeur esp\'er\'ee est $0$, c'est-\`a-dire, pas d'arbitrage. 

Supposons $0\notin C$, puisque $C$ est convexe, d'apr\`es le th\'eor\`eme de s\'eparation des convexes, alors 
\begin{equation}
\exists \alpha \in \mathbb{R}^d, \alpha^t \geq 0, \forall x\in C,\exists x_0\in C, \alpha^t x_0 > 0
\end{equation}
\begin{equation}
x_0=E^{Q_0}[\Delta\tilde{S}]
\end{equation}
Soit $\alpha_0$ arbitraire, on consid\`ere la strat\'egie $\theta=(\alpha_0, \alpha)$
On a
\begin{equation}
	\forall \theta\in\mathcal{E}, Q(\tilde{G}^\theta\geq 0)=1, Q_0(\tilde{G}^\theta>0)>0
\end{equation}
Et
\begin{equation}
	\forall Q\in\mathcal{E}, E^Q[\tilde{G}^Q]\geq 0, E^Q[\tilde{G}^Q]>0
\end{equation}

S'il existe $\omega_i$ tel que $G^Q(\omega_i)<0$, on va construire une mesure $Q^t$:
\begin{equation}
\begin{split}
	Q^\epsilon(\omega)= & 1-\frac{K-1}{K}\epsilon , \omega = \omega_i \\
	\frac{\epsilon}{K} & \omega\neq \omega_i
\end{split}
\end{equation}
on peut choisir $\epsilon$ suffisamment actif tel que $E^{Q_\epsilon}[\tilde{G}^{\theta}]<0$. Donc
\begin{equation}
	\forall i, G^\theta(\omega_i)\geq 0 \geq Q^0(G^\theta\geq 0)=1
\end{equation}
mais puisque $E^{Q_0}[\tilde{G}^\theta]>0$, alors $Q^0(\tilde{G}^\theta)>0$

$\Rightarrow$ Il y a arbitrage. Contradition, donc $0\in C \Rightarrow \exists $ une mesure de proba risque neutre.

\textbf{Exemple}:
\begin{equation}
\Omega=\{\omega_1, \omega_2,\omega_3, \omega_4\}, \mathbb{P}=\{0.3, 0.3,0.2,0.2\}
\end{equation}
$R=0.25$, $S^0$: Actif sans risque vaux $1$ en $t=0$.
$S^1$: une action qui vaut $1$ en $t=0$.
$S^2$: un call sur $S^1$ de strike $K=1$ qui vaut $0.3$ en $t=0$.
\begin{center}
\begin{tabular}{l||l|l|l|l|l}
		& $S_0^i$ & $w_1$ & $w_2$ & $w_3$ & $w_4$\\ \hline \hline
	$i=0$ & 1 & 1.25 & 1.25 & 1.25 & 1.25 \\	 \hline 
	$i=1$ & 1 & 0.5 & 1 & 1.5 & 2 \\ \hline 
	$i=2$ & 0.3 & 0 & 0 & 0.5 & 1 \\ 
\end{tabular}
\end{center}

Un actif vaut toujours la m\^{e}me valeur peu importe l'\'etat.

Exemple:
\begin{enumerate}
	\item Avec la strategie $\theta=(0.7, -1, 1)$, calculer la gain $G$ de $V^t$
	\item Est-ce qu'il y a une opportunit\'e d'arbitrage dans ce march\'e?
\end{enumerate}

\begin{equation}
G^\theta(\omega_i)=V_1(\omega_i)-V_0, V_0=1*0.7-1+0.3=0
\end{equation}

\begin{equation}
G^\theta(\omega_1)=0.375, G^\theta(\omega_{2,3,4})=-0.125
\end{equation}

Cherchons $Q=\{q_1,q_2,q_3,q_4\}$ tel que $0<q_i$ et
\begin{equation}
\left\{ \begin{array}{rcl}
	q_1+q_2+q_3+q_4 &= 1 \\
	\frac{0.5}{1.25} q_1+\frac{1}{1.25} q_2+\frac{1}{1.25} q_3 + \frac{2}{1.25} q_4 &= 1 \\
	\frac{0.5}{1.25} q_3 + \frac{1}{1.25} q_4 &=0.3
\end{array}\right.
\end{equation}

Et on a
\begin{equation}
\left\{ \begin{array}{rcl}
q_1&=\frac{1}{2}-2q_4 \\
q_2&=3q_4-\frac{1}{4} \\ 
q_3&=\frac{3}{4}-2q_4
\end{array}\right.
\end{equation}

l'ensemble des mesures risque neutre est :
\begin{equation}
\{(\frac{1}{2}-2\lambda,3\lambda-\frac{1}{4},\frac{3}{4}-2\lambda)|q_i>0,i\in\{1,2,3,4\}\}\neq\emptyset \Rightarrow AOA
\end{equation}

{\color{red} Is the above correct ?}

\section{March\'e complet et valorisation d'actif conditionnel}

On suppose que le march\'e est sans arbitrage.

\begin{defn}
	Un actif conditionnel est une variable al\'eatoire $H$ dans $(\Omega, \mathcal{F}, \mathbb{P})$.
\end{defn}

\begin{defn}
	Un actif conditionnel est r\'ealisable s'il existe une strat\'egie $\theta$ tel que $H=V_1^\theta$
\end{defn}

\begin{remq}
	Par AOA, on a prix de l'actif conditionnel $P(H)=V_0^\theta$.
\end{remq}

\begin{propos}
	Le juste prix d'un actif conditionnel r\'ealisable $H$ est:
	\begin{equation}
	P(H) = E^Q[\tilde{H}]=E^Q[\frac{H}{S_1^\theta}]=E^Q[\frac{H}{1+R}]
	\end{equation}
	o\`u $Q$ est une mesure de probabilit\'e risque neutre.
\end{propos}


\textbf{preuve}: 
$\theta$ est une strat\'egie tel que $V_1^\theta=H$
\begin{equation}
\begin{split}
E^Q[\tilde{H}]&=E^Q[\tilde{V}_1^\theta]\\
&=E^Q[\tilde{V}_0^\theta+\sum_{i=1}^d \theta_i \Delta \tilde{S}^i]\\
&=V_0^\theta+\sum_{i=1}^d \theta_i E^Q[\Delta \tilde{S}^i]\\
&=V_0^\theta=P(H)
\end{split}
\end{equation}

{\color{red} Pourquoi pas $V_0^\theta$ sur la 3e et 4e ligne?}

\begin{defn}
	Un march\'e est complet si tout actif conditionnel est r\'ealisable.
\end{defn}

\begin{remq}
	$A$: matrice des payoff: $A_{(i,k)}(S_1^i(w_k))$. Un actif conditionnel est r\'ealisable si $\exists \theta$ tel que $\theta^T A=H$. Et donc le march\'e est complet si $\theta^T A=H$ a une solution pour tout $H\in \mathbb{R}^K$.	
\end{remq}


Donc il faut que $A$ a au moins $K$ lignes lin\'eairement ind\'edendantes.[une condition n\'ecessaire est donc $d+1\geq K$]

\begin{thm}
	Un march\'e sans arbitrage est complet ssi il existe une seule mesure de probabilit\'e risque neutre.
\end{thm}

\textbf{Preuve}: Supposons que le march\'e est complet. Prenons l'actif conditionnel $\mathds{1}_{\{w_k\}}(1\leq k\leq K)$. Donc pour toute mesure de probabilit\'e risque neutre $q$ on a:
\begin{equation}
P_k=P(\mathds{1}_{\{w_k\}})=E^Q[\frac{\mathds{1}_{\{w_k\}}}{1+R}]=\frac{1}{1+R}Q(\{w_k\})
\end{equation}
$\Rightarrow$ Unicit\'e de $Q$. {\color{red} Pourquoi?}

Supposons que la mesure risque neutre est unique. D\'emontrer que le march\'e est complet.

Si le march\'e n'est pas complet 
\begin{equation}
	\exists z\in\mathbb{R}^{d+1}, z\neq 0, \text{t.q. } z^tA=0
\end{equation}
Soit
\begin{equation}
	Q_i|Q_2=\frac{Q(w_k)+\epsilon z_k}{1+\epsilon \sum z_k}
\end{equation}. 
Il suffit de prendre $\epsilon$ suffisamment petit, $0<Q_2(w_k)<1$.
\begin{equation}
	\begin{split}
		E^{Q_\epsilon}&=\frac{\sum_{k+1}^K (Q(w_k)+\epsilon z_k \tilde{S}^i(w_k)}{1+\epsilon \sum z_k}\\
		&=\sum_{k=1}^K Q(w_k) \tilde{S}_1^i(w_k)\times\epsilon \sum_k \mathds{1}^{K} z_k \tilde{S}_1^i (w_k)\\
		&=E^Q[\tilde{S}_1^i]\\ 
		&=\tilde{S}_0^i
	\end{split}
\end{equation}

\textbf{Exercise}
1. $\Omega=\{w_1, w_2\}$, on cherche le modele suivant:
\begin{equation}
	S_0^0=1 \rightarrow 1+\Omega(=10\%)
\end{equation}
\begin{equation}
	S_0^1=10\rightarrow\left\{
	\begin{array}{rc}
		12, &w_1,P_1\\
		11, &w_2,P_2
	\end{array}\right.
\end{equation}

Est-ce que le march\'e est sans abritrage? Si ce n'est pas le cas, mettre au cas une strat\'egie d'arbitrage.

Cherchons les mesures proba risque neutre
\begin{center}
\begin{tabular}{l|lll}
   & $S_0^1$ & $w_1$ & $w_2$ \\ \hline 
 $i=0$ & 1 & 1.1 & 1.1 \\
 $i=1$ & 10 & 12 & 11 \\
\end{tabular}	
\end{center}

Cherchons $P_1, P_2$ tel que 
\begin{equation}
P_1>0, P_2>0, E[\tilde{S_1}^0] = \tilde{S}_0^0, E[\tilde{S}_1^1]=\tilde{S}_0^1
\end{equation}
\begin{equation}
	P_1\frac{1.1}{1.1}+P_2\frac{1.1}{1.1}=\frac{1}{1} 
\end{equation}
and
\begin{equation}
	P_1\frac{12}{1.1}+P_2\frac{11}{1.1}=\frac{10}{1}
\end{equation}
Et on a
\begin{equation}
P_1=0, P_2=1
\end{equation}
 
$P_1>0$ n'est pas v\'erifi\'e, donc il existe opportunit\'e d'arbitrage.
On veut 10 de $S^0$ et on ach\`ete 1 de $S^1$. Ainsi, la valeur initiale du portefeuille est
\begin{equation}
	V_0=-10S_0^0+S_0^1=0
\end{equation}
Donc au $t=1$
\begin{equation}
V_1\rightarrow
\left\{\begin{array}{rc}
	-10(1+R)+12=1\\
	-10(1+R)+11=0
\end{array}\right.
\end{equation}

2. $\Omega=\{w_1, w_2\}$, on cherche le modele suivant:
\begin{equation}
	S_0^0=1 \rightarrow 1+\Omega(=10\%)
\end{equation}
\begin{equation}
	S_0^1=10\rightarrow
	\left\{\begin{array}{rc}
		12, w_1, P_1\\
		8, w_2, P_2
	\end{array}\right.
\end{equation}

\begin{enumerate}
\item Est-ce que le marche est sans arbitrage(complet)?
\item Calculer le prix d'un call de strike 10
\item Calculer de deux fa\c{c}ons le prix d'un put de strike 10
\end{enumerate}

Solution:

1) $P_1=\frac{3}{4}, P_2=\frac{1}{4}$, donc il existe unique mesure probabilite risque neutre. Donc le march\'e est sans arbitrage(complet).

2) Puisque le march\'e est complet et sans arbitrage, alors:
\begin{equation}
\begin{split}
	Call_0(1,10) &= E^Q[\tilde{H}] \\
	&= E^Q[\frac{V_1^\theta}{1+R}] \\
	&= E^Q[\frac{(S_1^1-10)_+}{1+R}]\\
	&=\frac{\frac{3}{4}*2+\frac{1}{4}*0}{1.1}=\frac{15}{11}
\end{split}
\end{equation}

3) 
D'abord, on utilise la parit\'e Call-Put
\begin{equation}
c(1,10)-P(1,10)=S_0^1-\frac{K}{1+R}=\frac{10}{11}
\end{equation}
\begin{equation}
\Rightarrow P(1,10)=\frac{5}{11}
\end{equation}

\begin{equation}
P(1,10) = E^Q[\frac{(10-S_1^1)_+}{1+R}]=0\times\frac{3}{4}+\frac{2}{1.1}\times\frac{1}{4}=\frac{5}{11}
\end{equation}

\section{Le mod\`ele binominale \`a une p\'eriode}

$T=1,\Omega=\{w_u,w_d\},F=\mathbb{P}(\Omega)$,$\mathbb{P}$ mesure de probabilit\'e dans $(\Omega,\mathcal{F}),(0<P(w_u)<1)$

\begin{equation}
S_0^0=1\rightarrow 1+R
\end{equation}
\begin{equation}
S_0^1=s\rightarrow
\left\{\begin{array}{rc}
	us, &w_u,P_1\\
	ds, &w_d, P_2
\end{array}\right.
\end{equation}
o\`u $u \geq d$

\begin{propos}
	Le mod\`ele est viable (sans arbitrage), ssi
	\begin{equation}
	d<1+R<u
	\end{equation}
	(condition de non-arbitrage)
\end{propos}

\begin{propos}
	Le prix d'un actif conditionnel $H$ est:
	\begin{equation}
P(H)=E^Q[\frac{H}{1+R}]
\end{equation} 
o\`u
\begin{equation}
Q(w) = \left\{\begin{array}{rcl}
	\frac{1+R-d}{u-d} & ,\text{si} &w=w_d\\
	\frac{u-1-R}{u-d} & ,\text{si} &w=w_u
\end{array} \right.
\end{equation}
En plus, le portefeuille de couverture est donn\'ee par $(\theta_0,\theta_1)$ o\`u 
\begin{equation}
	\left\{\begin{array}{rc}
		\theta_0=P(H)-\theta_1 s\\
		\theta_1=\frac{H(w_u)-H(w_d)}{us-ds}
	\end{array}\right.
\end{equation}
\end{propos}

\textbf{Preuve}
\begin{enumerate}
\item Cherchons des mesures de probabilit\'e risque neutre telles que
\begin{equation}
	Q=\{q_u,q_d\}(q_u=Q(w_u),q_d=Q(w_d))
\end{equation}
\begin{equation}
\left\{ \begin{array}{rcl}
q_u+q_d&=1\\
\frac{q_uus+q_d ds}{1+R}&=s
\end{array}\right. \Leftrightarrow
\left\{\begin{array}{rc}
q_d&=1-q_u\\
q_us(u-d)&=s(1+R)-ds
\end{array}\right.
\end{equation}
Ainsi\begin{equation}
\left\{\begin{array}{rc}
	q_u=\frac{u-1-R}{u-d}\\
	q_d=\frac{1+R-d}{u-d}
\end{array}\right.
\end{equation}

Donc $\exists !$ mesure de probabilit\'e risque neutre ssi $d<1+R<u$

\item Si le march\'e est sans arbitrage et complet
\begin{equation}
	P(H)=E^Q[\frac{H}{1+R}]
\end{equation}
Cherchons $(\theta_0,\theta_1)$ tel que $\theta_0S_1^0 + \theta_1S_1^1=H$
\begin{equation}
\left\{\begin{array}{rcl}
\theta_0(1+R)+\theta_1 us&=H(w_u)\\
\theta_0(1+R)+\theta_1 ds&=H(w_d)
\end{array}\right.
\end{equation}
Ainsi
\begin{equation}
	\theta_1(us-ds)=H(w_u)-H(w_d)
\end{equation}
\begin{equation}
	\theta_1=\frac{H(w_u)-H(w_d)}{us-ds}
\end{equation}
Au temps $t=0$, on a
\begin{equation}
	\theta_0\times 1+\theta_1 s=P(H)
\end{equation}
Donc,
\begin{equation}
	\theta_0=P(H)-\theta_1 s
\end{equation}
La strat\'egie de couverture est not\'e $(\theta_0,\theta_1)$ ou $(P(H),\theta_1)$
qui donne le prix \`a definir pour l'acheter si le vendeur veut eliminer tout risque.
\end{enumerate}

\section{Th\'eor\`em de Valorisation dans le cas discr\`ete multi-p\'eriode}
\subsection{Mod\`ele}
On consid\`ere qu'il y a $n$ p\'eriodes et $n+1$ dates ${t_0=0,t_1,\ldots,t_n=T}$. L'al\'ea est repr\'esent\'e par un espace de probabilit\'e $(\Omega,\mathcal{F},P)$. Pour sp\'ecifier l'information disponible en $t$, on introduit la filtration $\mathbb{F}=\{\mathcal{F}_0,\ldots,\mathcal{F}_n\}$. 

On suppose $\mathcal{F}_0=\{\emptyset,\Omega\},\mathcal{F}_n=\mathcal{F}=P(\Omega)$. Les actifs risqu\'es sont mod\'elis\'es par $d$ processus stochastiques $(S^i_j)_{j=0,\ldots,n}^{i=1,\ldots,d}$ comme $\mathbb{F}$-adapt\'e. L'actif sans risque est d\'efini $(S_j^0)_{j=0,\ldots,n}$
\begin{equation}
S_0^0=1,S_j^0=(1+R_j)S_{j-1}^0
\end{equation} 
$R_j$ est le taux d'int\'er\^et pour la p\'eriode $[t_{j-1},t^j]$.

\begin{defn}
	Une strat\'egie d'investissement admissible est un processus stochastique $(\theta_j^i)_{j=0,\ldots,n}^{i=0,\ldots,d}$ o\`u $\theta_j^i$ est la quantit\'e investie dans l'actif $i$ entre $t_{j-1}$ et $t_j$ qui doit \^etre connu en $t_{j-1}$.($\theta$ est pr\'evisible).
\end{defn}

\begin{remq}
	La valeur d'un portefeuille peut varier par deux possibilit\'es
	\begin{enumerate}
		\item variation des prix d'actifs
		\item versement/retrait d'argent
	\end{enumerate}
\end{remq}

On va introduire la possibilit\'e 2 aux strat\'egies. 

\begin{defn}
	Une strat\'egie $\theta$ est dite auto-financ\'ee si 
\begin{equation}
\sum_{i=0}^d\theta_j^i S_j^i=\sum_{i=0}^d\theta_{j-1}^i S_j^i, \forall j=1,\ldots,n-1
\end{equation}
o\`u le changement de la strat\'egie d'investissement ne change pas la valeur totale du portefeuille pour un instant donn\'e.
\end{defn}

\begin{defn}
	Le processus de gain d'une strat\'egie est un processus stochastique qui d\'ecrit le changement de valeur du portefeuille associ\'e suite aux variation des prix d'actif.

\end{defn}

On va le noter $G$
\begin{equation}
G_0=0,G_t=\sum_{j=1}^t\sum_{i=0}^d \theta_j^i\Delta S_j^i,\Delta S_j^i=S_j^i-S_{j-1}^i
\end{equation}
pour une strat\'egie auto-financ\'ee,
\begin{equation}
V_t=V_0+G_t
\end{equation}
Comme dans le cas d'une p\'eriode, on note:
\begin{equation}
\tilde{S}_j^i=\frac{S_t^i}{S_t^0},\tilde{V}_t=\frac{V_t}{S_t^0}=\tilde{V}_0+\tilde{G}_t
\end{equation}
\begin{equation}
\tilde{G}_t=\sum_{j=1}^t\sum_{i=1}^d \theta_j^i \Delta \tilde{S}_j^i
\end{equation}

\subsection{R\'esultats du march\'e viable et complet}

\begin{defn}
	Une opportunit\'e d'arbitrage est une strat\'egie auto-financ\'ee $\theta$ tel que:
\begin{equation}
\left\{\begin{array}{l}
V_0^\theta=0 \\
P(V_n^\theta \geq 0)=1\\ 
P(V_n^\theta >0)>0
\end{array}\right.
\end{equation}
ou
\begin{equation}
\left\{\begin{array}{rcl}
P(\tilde{G}_n^\theta \geq 0)=1\\
P(\tilde{G}_n^\theta>0)>0
\end{array}\right.
\end{equation}
\end{defn}

\begin{thm}
	Un march\'e financ\'e \`a $n$ p\'eriode est viable ssi il existe une mesure de probabilit\'e risque neutre $Q$. C'est-\`a-dire,
\begin{equation}
\left\{\begin{array}{l}
Q(w)>0,\forall w\in\Omega\\
E^Q[\tilde{S_{t+1}^i}|\mathcal{F}_t]=\tilde{S}_t^i
\end{array}\right.
\forall 0 \leq i \leq d,0\leq t\leq n-1
\end{equation}
(Les prix actualis\'es sont de $Q$-martingale).
\end{thm}

{\color{red} C'est quoi martingale?}

\begin{propos}
	Le juste prix d'un actif conditionnel r\'ealisable $H$ \`a la date $t$ est(l'actif verse $H$ en $T$):
\begin{equation}
P_t(H)=S_t^0 E^Q[\frac{H}{S^0_t}|\mathcal{F}_t]
\end{equation}
($Q$ est la mesure risque neutre)
\end{propos}

\textbf{Preuve}:
Il existe une strat\'egie $\theta$ tel que 
\begin{equation}
H=\sum_{i=0}^d \theta_i S_t^i
\end{equation}
Donc
\begin{equation}
\begin{split}
S_tE^Q[\frac{H}{S_n^0}|\mathcal{F}_t]&=S_tE^Q[\frac{\sum\theta_i S_n^i}{S_n^0}|\mathcal{F}_t]\\
&=S_t\sum_{i=0}^d\theta_i E^Q[\tilde{S}_n^i|\mathcal{F}_t]\\
&=S_t\sum_{i=0}^d\theta_i\tilde{S}_t^i\\
&=\sum_{i=0}^d\theta_i S_t^i=V_t=P_t(H)
\end{split}
\end{equation}

{\color{red} Pourquoi $E^Q$ disparait sur la 3e ligne?}

\begin{thm}
	Un march\'e viable est complet ssi la moyenne de probabilit\'e risque neutre est unique.
\end{thm}


\subsection{Mod\`ele binomiale multi-p\'eriode(Mod\`ele de Lex-Ross-Robinstein)}
\begin{equation}
\Omega=\{-1,1\}^n,\mathcal{F}=P(\Omega)
\end{equation}
Soit $(z_k)_{k>=0}$ une suite de variable al\'eatoires ind\'ependantes telles que $P(z_k=1)=P(z_k=-1)=\frac{1}{2}$
\begin{equation}
\mathcal{F}_0=\{\emptyset,\Omega\},\mathcal{F}_k=\sigma(z_0,\ldots,z_k),\mathbb{F}=\{\mathcal{F}_0,\ldots,\mathcal{F}_n\}
\end{equation}

$T>0, t_i=\frac{iT}{n}$: diviser $[0,T]$ en $n$ parties.

On d\'efinit:
\begin{equation}
\left\{\begin{array}{rcl}
b_n=b\frac{T}{n}\\
\sigma_n=\sigma\frac{T}{n}
\end{array}\right.
,b,\sigma>0
\end{equation}

\begin{equation}
S_0^0=1, S_j^0=e^{\Omega\frac{T}{n}}S_{j-1}^0,(R=e^{\Omega\frac{T}{n}}-1 \frac{tT}{n})=(1+R)S_{j-1}^0
\end{equation}
{\color{red} Comprends pas...}
\begin{equation}
\begin{split}
S_0^1=s,S_j^1&=se^{jb_n+\sigma_n\sum{k=1}^j z_k}\\
&=S_{j-1}^1 e^{b_n+\sigma_n z_j}, j=1,\ldots,n
\end{split}
\end{equation}

Ainsi, pour chaque $n>=1$, on a definit un march\'e \`a $n$ p\'eriodes de pas constant $\frac{T}{n}$.

\subsubsection{Proposition}
Le march\'e est sans arbitrage ssi
\begin{equation}
d_n<1+R_n<u_n  
\end{equation}
avec
\begin{equation}
	\left\{\begin{array}{rcl}
u_n=e^{b_n+\sigma_n}\\
d_n=e^{b_n-\sigma_n}
\end{array}\right.
\end{equation}

Dans ce cas, il existe une unique mesure de probabilit\'e risque neutre $Q^n$ d\'efinie par:
\begin{equation}
Q^n(z_i=1)=q_n=\frac{1+R_n-d_n}{u_n-d_n}
\end{equation}
En plus, le march\'e est complet.
\subsection{Valorisation dans le mod\`ele de Cor-Ross-Robinstein}
On consid\`ere un actif conditionnel $H_n=g(S_n^1)$
\begin{equation}
P_0(H_n)=S_0^0E^{Q^n}[\frac{H_n}{S_n^0}]
\end{equation}
\begin{equation}
\left\{\begin{array}{l}
	S_0^0=1\\
	S_n^0=(1+R)S_{n-1}^0=\ldots=(1+R)^n=(e^{R\frac{T}{n}})^n=e^{nT}
\end{array}\right.
\end{equation}
\begin{equation}
\begin{split}
	P_0(H_n)&=\frac{1}{(1+R_n)^n}E^{Q_n}[H_n]\\
	&=e^{-R T}E^{Q_n}[H_n]\\
	&=e^{-R T}E^{Q_n}[g(S_n^1)]
\end{split}
\end{equation}
{\color{red} C'est quoi $g()$ ?}

Sous $Q^n$, $(\frac{z_{j+1}}{2})$ est une suite de V.A. de Bernouille de param\`etre $q_n=\frac{1+R_n-d_n}{u_n-d_N}$

Valeur moyenne de Bernouille:
\begin{equation}
Q^n(\sum_{i=1}{n}\frac{1+z_i}{2}=j)=C_n^j(q_n)^j(1-q_n)^{n-j},(0\leq j\leq n)
\end{equation}
\begin{equation}
P_0(H_n)=e^{-RT}\sum_{i=0}^n C_n^i(q_n)^i(1-q_n)^{n-i}g(i(u_n)^i(d_n)^{n-i})
\end{equation}
\begin{equation}
P_k(H_n)=e(-R(1-\frac{k}{n})T)\sum_{i=0}^{n-k}C_n^iq_n^i(1-q_n)^{n-k-i}g(S_k^iu_n^id_n^{n-k-i})
\end{equation}

\begin{propos}
	\begin{equation}
\begin{split}
P_k(H_n)&=e^{-R\frac{T}{n}}E^{Q_n}[P_{k+1}(H_n)|\mathcal{F}_k]\\
&=e^{-\frac{RT}{n}}[q_nP_{k+1}(H_n)_{u_n}+(1-q_n)P_{k+1}(H_n)_{d_n}]
\end{split}
\end{equation}
o\`u $,k=0,\ldots,n-1$
\begin{equation}
\begin{split}
E^{Q_n}[P_{k+1}(H_n)|\mathcal{F}_k]&=E^{Q_n}[S_{k+1}^0E^{Q_n}[\frac{H_n}{S_n^0}|\mathcal{F}_{k+1}]|\mathcal{F}_k]\\
&=S^0_{k+1}E^{Q_n}[\frac{H_n}{S_n^0}|\mathcal{F}_k]\\
&=e^{R\frac{T}{n}}P_k[H_n]
\end{split}
\end{equation}
\end{propos}


\textbf{Remarque}:
La valorisation et la couverture dans le cas multi-p\'eriode se r\'eduit \`a une s\'equence de valorisation/couverture dans le cas d'une seule p\'eriode.

En effet, $P_k(H_n)$ est le juste prix de l'actif conditionnel $P_{k+1}(H_n)$.
\begin{center}
\begin{tabular}{ll}
 $\frac{kT}{n}$ & $\frac{(k+1)T}{n}$\\
 Actif risqu\'e: $S_k^1$ & $\rightarrow u_nS_k^1$\\
 & $\rightarrow d_nS_k^n$
\end{tabular}
\end{center}

Actif conditionnel:
\begin{center}
\begin{tabular}{ll}
$\frac{kT}{n}$ & $\frac{(k-1)T}{n}$ \\ 
$P_k(H_n)$ & $\rightarrow P_{k+1}(H_n)_{u_b}$\\
&$\rightarrow P_{k+1}(H_n)_{d_n}$
\end{tabular}
\end{center}

La strat\'egie de couverture s'\'ecrit donc
\begin{equation}
\theta_{k+1}^1=\frac{P_{k+1}{(H_n)_{u_n}}-P_{k+1}(H_n)_{d_n}}{u_nS_k^i-d_nS_k^i}
\end{equation}
On la tracte $(P_k(H_n),\theta_{k+1}^1)$.

\subsubsection{Exercise}
On consid\`re le mod\`ele suivant:
\begin{equation}
S_0^0=1\rightarrow S_1^0=1+R\rightarrow S_2^0=(1+R)^2
\end{equation}
\begin{equation}
S_0^1=2
\left\{
\begin{array}{l}
\rightarrow 4\left\{\begin{array}{l}
8\\
2
\end{array}\right.\\
\rightarrow 1
\left\{\begin{array}{l}
2\\
\frac{1}{2}
\end{array}\right.
\end{array}\right.
\end{equation}

1) Est-ce que le march\'e est viable/complet?

2) Calculer le prix d'un call sur l'actif risqu\'e $S^1$ de maturit\'e $2$ et de strike $1$.

Solution:

1)
\begin{equation}
u_1=\frac{4}{2}=u_2=\frac{8}{4}=2
\end{equation}
\begin{equation}
	d_1=\frac{1}{2}=d_2=\frac{\frac{1}{2}}{1}=\frac{1}{2}
\end{equation}
\begin{equation}
d_i=\frac{1}{2}<1+R=1.25<2=u_i
\end{equation}
Ainsi, le march\'e est viable car la condition $d<1+R<u$ est satisfaite.

\textbf{Note}:

March\'e viable $\rightarrow$ AOA

March\'e complet:

Toute option peut \^etre couverte (r\'epliqu\'ee) parfaitement par un portefeuille auto-financ\'e. 

2)
\begin{equation}
P_k(H_n)=\frac{1}{1+R}[q_kP_{k+1}(H_n)+(1-q_n)P_{k}(H_n)]e^{-R}\approx=\frac{1}{1+R}
\end{equation}

$Q_k=\{q_k,1-q_k\}$ est la mesure de proba risque neutre.
\begin{equation}
P_2(H_2)=
\left\{\begin{array}{r}
7\\
1\\
1\\
0
\end{array}\right.
\end{equation}
\begin{equation}
H_2=(S_2^1-1)_+
\end{equation}
pour $S_1^1=4$:
\begin{equation}
q_2=\frac{1.25-0.5}{2-0.5}=\frac{0.75}{1.5}\frac{1}{2}
\end{equation}
\begin{equation}
P_1(H_2)=\frac{1}{1.25}[\frac{1}{2}\times 7+\frac{1}{2}\times 1]=\frac{16}{5}
\end{equation}
pour $S_1^1=1$:$q_2=\frac{1}{2}$
\begin{equation}
P_1(H_2)=\frac{4}{5}\times[\frac{1}{2}\times 1+\frac{1}{2}\times 0]=\frac{2}{5}
\end{equation}
et en $t=0$: $q_1=\frac{1}{2}$
\begin{equation}
P_0(H_2)=\frac{4}{5}\times[\frac{1}{2}\times\frac{16}{5}+\frac{1}{2}\times\frac{2}{5}]=\frac{36}{25}
\end{equation}

\begin{thm}
	Le juste prix $P_0(H_n)$ $(H_n=(S_n^1-K)_+)$ converge quand $n->\infty$ au prix Black Scholes.
\end{thm}
\begin{equation}
P_0(S,T,K)=S\mathcal{N}(d_1(S,K,\sigma^2 T))-Ke^{R T}\mathcal{N}(d_2(S,K,\sigma^2 T))
\end{equation}
\begin{equation}
d_1(S,K,v)=\frac{ln(\frac{S}{Ke^{R t}})}{Nv}+\frac{1}{v}Sv
\end{equation}
\begin{equation}
d_2(S,K,v)=d_1-\frac{1}{2}Sv
\end{equation}
\begin{equation}
\mathcal{N}(x)=\int_{-\infty}^x\frac{e^{-\frac{y^2}{2}}}{\sqrt{2\pi}}dy
\end{equation}
est la fonction de r\'epartition de la loi normale standard.

\section{Valorisation dans un cadre continue: Le mod\`ele de Black-Scholes}
\subsection{Introduction}
Notre objectif est de donner dans un mod\`ele probabiliste continu un prix \`a une option et g\'en\'eralement \`a tout contrat de flux terminale en $T$ de la forme $g(S_T)$ avec $S_t$ un titre n\'egociable.

Nous traitons dans ce chap\^itre un mod\`ele de base en finance, qui est le mod\`ele de Black Scholes.

\subsection{Mod\`ele de Black Scholes}
L'al\'ea du march\'e financier est mod\'elis\'e via un espace de probabilit\'e filtr\'e $(\Omega,\mathcal{F}_1(\mathcal{F}_t),\mathbb{P}),0\leq t\leq T$

$\Omega$: L'ensemble des scenarii possibles
$\mathcal{F}$: est une tribu qui r\'epr\'esente l'information globale disponible sur le march\'e.

Les al\'eas sont g\'en\'er\'es par un mouvement brownien standard $(W_t)_{0\leq t\leq T}$ qui engendre le filtration $\mathbb{F}=(\mathcal{F})_{0\leq t\leq T}$

$\mathbb{P}$ est la probabilit\'e historique.

\textbf{Rappel}:

Un mouvement brownien standard est un processus stochastique $(W_t)_{t>0}$
\begin{equation}
\left\{\begin{array}{l}
W_0=0 \text{(standard) et } W_t \text{ est continu}\\
\text{accroissements ind\'ependants}:\\
\forall t_1\leq\ldots \leq t_n, (W_{t_n}-W_{t_{n-1}}),\ldots,(W_{t_2}-W_{t_1}) \text{ sont ind\'ependants.}\\
W_t-W_s \text{ est de loi } \mathcal{N}(0,t-s),\forall t>s\\
\end{array}\right.
\end{equation}

\begin{defn}
	Le mod\`ele de Black \& Scholes est d\'efini sous forme de rendement instantan\'e par
\begin{equation}
\frac{d S_t}{S_t}=b dt+\sigma d W_t
\end{equation}
\begin{equation}
	d S_t=b S_t dt+\sigma S_t d W_t
\end{equation}
\begin{equation}
S_t=S_0+\int_0^t S_u du +\int_0^t \sigma S_u dW_u
\end{equation}
\end{defn}


\begin{equation}
\int_0^t X_u dW_u=\lim_{\Delta t\rightarrow 0}\sum_{i=1}^{n}X_{t_i}(W_{t+1}-W_t)
\end{equation}

avec $S_0=x$ et $W_t$ est un M.B.S(Mouvement Brownien Stochastique) sous $\mathbb{P}$
Ainsi
\begin{equation}
S_t=S_0e^{(b-\frac{1}{2}\sigma^2)t+\sigma W_t}
\end{equation}
On suppose que l'on a un actif sans risque $S^0$ dont la valeur $S_t^0=e^{Rt}$
\begin{remq}
	D\'efinissons le rendement de l'actif $S$ en $t=1$ et $t_i$
\end{remq}
\begin{equation}
t_i(\Delta t_i=t_i-t_{i-1})
\end{equation}
\begin{equation}
\begin{split}
R_{t_i}&=\frac{S_{t_i}-S_{t_{i-1}}}{S_{t_{i-1}}}=\frac{S_{t_{i-1}}e^{(b-\frac{1}{2}\sigma^2)\Delta t_i+\sigma(W_{t_i}-W_{t_{i-1}})}-S_{t_{i-1}}}{S_{t_{i-1}}}\\
&=e^{(b-\frac{1}{2}\sigma^2)\Delta t_i+\sigma(W_{t_i}-W_{t_{i-1}})}-1
\end{split}
\end{equation}
Esp\'erance du rendement:
\begin{equation}
\begin{split}
E[R_{t-1}]&=E[e^{(b-\frac{1}{2}\sigma^2)\Delta t_i+\sigma(W_{t_i}-W_{t_{i-1}})}-1]\\
&=e^{(b-\frac{1}{2}\sigma^2)\Delta t_i} E[e^{\sigma(W_{t_i}-W_{t_{i-1}})}]-1\\
&=e^{(b-\frac{1}{2}\sigma^2)\Delta t_i}\times e^{\frac{1}{2}\sigma^2\Delta t_i}-1\\
&=e^{b\Delta t_i}-1\approx b\Delta t_i+o(\Delta t_i^2)
\end{split}
\end{equation}
Variance du rendement:
\begin{equation}
\begin{split}
Var(R_{t_i})&=E[(R_{t_i}-E[R_{t_i}])^2]\\
&=E[(e^{(b-\frac{1}{2}\sigma^2)\Delta t_i+\sigma(W_{t_i}-W_{t_{i-1}})}-e^{b\Delta t_i})^2]\\
&=e^{2b\Delta t_i}E[(e^{-\frac{1}{2}\sigma^2\Delta t_i+\sigma(W_{t_i}-W_{t_{i-1}})}-1)^2]\\
&=e^{2b\Delta t_i}(e^{-\sigma^2\Delta t_i}E[e^{2\sigma (W_{t_i}-W_{t_{i-1}})}]-2e^{-\frac{1}{2}\sigma^2\Delta t_i}E[e^{\sigma (W_{t_i}-W_{t_{i-1}})}]+1)\\
&=e^{2b\Delta t_i}(e^{-\sigma^2\Delta t_i} e^{\frac{1}{2}(2\sigma)^2\Delta t_i}-2e^{-\frac{1}{2}\sigma^2\Delta t_i}e^{\frac{1}{2}\sigma^2\Delta t_i}+1)\\
&=e^{2b\Delta t_i}(e^{\sigma^2\Delta t_i}-1)\\
&\approx\sigma^2\Delta t_i+o(\Delta t_i^2)
\end{split}
\end{equation}
Ainsi,
\begin{equation}
\sigma\approx\sqrt{\frac{Var(R_{t_i})}{\Delta t_i}}
\end{equation}
$\sigma$ est donc l'\'ecart type normalis\'e du rendement, appll\'ee volatilit\'e.
$b$ est l'esp\'erance du standard (normalis\'e) sur une courte p\'eriode $\Delta t_i$.

\begin{equation}
dS_t=bS_tdt+\sigma S_t dW_t \Rightarrow S_t=S_0 e^{(b-\frac{1}{2}\sigma^2 t)+\sigma W_t}
\end{equation}
\begin{equation}
\begin{split}
ln(S_t)&=ln(S_0)+\int_0^u (\frac{\partial ln}{\partial S}(S_t) dS_t)+\int_0^u \frac{1}{2}(\frac{\partial^2 ln}{\partial S^2}(S_t)dS_t)\\
&=ln(S_0)+\int_0^u\frac{1}{S_t} dS_t -\int_0^u \frac{1}{S_t^2}(\sigma S_t)^2 dt\\
&=ln(S_0)-\frac{1}{2}\int_0^u\sigma^2 dt + \int_0^u \frac{1}{S_t} (b_t S_t+dt)+\int_0^u \frac{1}{S_t}\sigma S_t dW_t\\
&=ln(S_0)-\frac{1}{2}\sigma^2 u+b u+\int_0^u\sigma d W_t\\
&=ln(S_0)+(b-\frac{1}{2}\sigma^2)u+\sigma W_u
\end{split}
\end{equation}
Ainsi,
\begin{equation}
S_u=S_0e^{(b-\frac{1}{2}\sigma^2)u+\sigma W_u}
\end{equation}
Supposons que le prix d'une option en $t$ est $u(t,S_t)=P_t$
\begin{equation}
\begin{split}
dP_t&=du(t,S_t)\\
&=\frac{\partial u}{\partial t}(t, S_t)dt+\frac{\partial u}{\partial S}(t,S_t)dS_t+\frac{1}{2}\frac{\partial^2 u}{\partial S^2}(t,S_t)dS_t\\
&=(\frac{\partial u}{\partial t}(t,S_t)+\frac{1}{2}\sigma^2S_t^2\frac{\partial^2}{\partial S^2}(t, S_t))dt+\frac{\partial u}{\partial S}(t, S_t)(bS_tdt + \sigma S_tdW_t)\\
&=(\frac{\partial u}{\partial t}+\frac{1}{2}\sigma^2 S_t^2\frac{\partial^2 u}{\partial S^2}+bS_t\frac{\partial u}{\partial S})dt+\sigma S_t\frac{\partial u}{\partial S}dW_t
\end{split}
\end{equation}
Ainsi la volatilit\'e de l'option en $t$ est:
\begin{equation}
\frac{\sigma S_t \frac{\partial u}{\partial S}(t, S_t)}{u(t, S_t)}
\end{equation}

\subsection{Dynamique d'un portefeuille auto-finan\c{c}ant}
On consid\`ere un portefeuille financier contenant une centaine $S_t$ de l'actif risqu\'e $S$ et le reste dans l'actif sans risque $S^0$.
On suppose que ce portefeuille est auto-finan\c{c}ant.(Il n'y a pas de virement d'argent ou de retrait d'argent). Soit $V_t$
 sa valeur en $t$.
 
Si l'ajustement est fait \`a des dates discr\`ere $t_1,\ldots,t_n$,  alors
\begin{equation}
\begin{split}
V_{t_{i+1}}-V_{t_i}&=\delta_{t_i}(S_{t_i+1}-S_{t_i})+\frac{V_{t_i}-\delta_{t_i}S_{t_i}}{S^2_{t_i}}(S^0_{t_{i+1}}-S^0_{t_{i}})\\
&=\delta_{t_i}(S_{t_{i+1}}-S_{t_i})+(V_{t_i}-\delta_{t_i}S_{t_i})(e^{R(t_{i+1}-t_i)}-1)
\end{split}
\end{equation}
{\color{red} Pourquoi ?}
\begin{equation}
\tilde{V}_{t_{i+1}}-\tilde{V}_{t_i}=\delta_{t_i}(\tilde{S}_{t_{i+1}}-\tilde{S}_{t_i})
\end{equation}
En faisant tendre le pas d'ajustement $\Delta t_i=t_{i+1}-t_i$ vers $0$, on obtient
\begin{equation}
d\tilde{V}_t=\delta_td\tilde{S}_t
\end{equation}
\begin{equation}
dV_t=\delta_tdS_t+(V_t-\delta_t S_t)Rdt
\end{equation}
ce qui est l'\'equation du portefeuille auto-finan\c{c}ant en temps continue. 

\subsection{Construction du portefeuille de replication}

Rapplons que notre objectif est de valoriser ainsi que de couvrir une option de payoff $g(S_T)$ \`a \'ech\'eance $T$.
Soit $v(t,S):[0,T]\times]0,\infty[ $ une fonction dans $C^{1,2}$ 
 
Nous cherchons les conditions que doit v\'erifier $v$ pour qu'elle soit le prix de l'option.
D'abord, nous cherchons un portefeuille auto-finan\c{c}ant $V$ qui r\'eplique $v$, c'est-\`a-dire
\begin{equation}
V_t=v(t,S_t)
\end{equation}
en tout t.
\begin{equation}
f(t,X_t)=f(0,X_0)+\int_0^tf'(u,X_u)dX_u+\int_0^tf''(u,X_u)dX_u
\end{equation}
Rappel
\begin{equation}
dX_t=a_tdt+b_tdW_t
\end{equation}
\begin{equation}
dY_t=c_tdt+d_tdW_t
\end{equation}
\begin{equation}
d<X,Y>_t=(b_td_t)dt
\end{equation}
\begin{equation}
d<X>_t=d<X,X>_t=b_t^2dt
\end{equation}

\begin{equation}
\begin{split}
v(t, S_t)&=v(0,S_0)+\int_0^t\frac{\partial v}{\partial S}(u,S_u)du+\int_0^t\frac{\partial v}{\partial S}(u, S_u)
 dS_u + \frac{1}{2}\int_0^t\frac{\partial^2 v}{\partial S^2}(u,S_u)dS_u\\
 &=v(0,S_0)+\int_0^t\frac{\partial v}{\partial u}(u,S_u)du+\frac{1}{2}\int_0^t\sigma^2S_u^2\frac{\partial^2 v}{\partial S}(u,S_u)du+\int_0^t\frac{\partial v}{\partial S}(u, S_u)dS_u\\
 &=v(0,S_0)+\int (\frac{\partial v}{\partial u}(u,S_u)+\frac{1}{2}\sigma^2S_u^2\frac{\partial^2 v}{\partial S^2}(u,S_u))du+\int \frac{\partial v}{\partial S}(u, S_u)dS_u
\end{split}
\end{equation}
\begin{equation}
dV_t=S_tdS_t+(V_t-\delta_tS_t)r dt
\end{equation}
\begin{equation}
V_t=V_0+\int S_u dS_u+\int (V_u-\delta_uS_u)rdu
\end{equation}

Prenons $v$ solution de l'EDP,
\begin{equation}
\frac{\partial v}{\partial t}(t,S)+\frac{1}{2}\sigma^2S^2\frac{\partial^2 v}{\partial S^2}(t,,S)=(v(t,S_t)-\frac{\partial v}{\partial S}(u,S)S)R
\end{equation}
Donc 
\begin{equation}
v(t,S_t)=v(0,S_0)+\int\frac{\partial v}{\partial S}(u,S_u) dS_u+\int (v(u,S_u)-S_u\frac{\partial v}{\partial S})Rdt
\end{equation}
Si je choisis
\begin{equation}
V_0=v(0,S_0), S_t=\frac{\partial v}{\partial S}(t,S_t)
\end{equation}
alors
\begin{equation}
V_t=v(t,S_t)
\end{equation}
en tout $t$ et en particulier $V_T=v(T,S_T)$

Pour A.O.A. On a

valeur de l'option = valeur du portefeuille de couverture

\begin{thm}
	Dans le mod\`ele de Black Scholes, le prix en $t$ d'une option europ\'eenne de payoff $g(S_T)$ en $T$ est $v(t,S_t)$ o\`u la fonction $v$ est la solution de l'EDP:
	\begin{equation}
\frac{\partial v}{\partial t}+\frac{1}{2} \sigma^2 S^2 \frac{\partial^2 v}{\partial S^2}=r(v-S\frac{\partial v}{\partial S})
\end{equation}
\begin{equation}
v(T,S) = g(S)
\end{equation}
Cette option peut-\^etrte couverte par un portefeuille auto-finan\c{c}ant de valeur initiale $v(0,S_0)=V_0$ et qui contient \`a tout $t$ $\frac{\partial v}{\partial S}(t,S_t)$ d'unit\'e d'actif risqu\'e.
\end{thm}

\begin{remq}
	Pour r\'epliquer une option europ\'eenne dans le mod\'ele B\&S, on peut utiliser la proc\'edure suivante:
\end{remq}
\begin{enumerate}
\item Calculer la fonction $v(t,S)$ r\'esolvant l'EDP

\item Calculer la d\'eriv\'ee $\frac{\partial v}{\partial S}$ pour obtenir le ratio de couverture.
\end{enumerate}

Pour que cette d\'emarche fonctionne, il faut que l'EDP admet une solution unique. D'o\'u le th\'eor\`eme suivant :

\begin{thm}
	Soit $g$ une fonction \`a croissance polynomiale.
\begin{equation}
\exists p, \forall x, |g(x)|<=c(1+|x|^p)
\end{equation}
Alors l'EDP de B\&S admet une solution dans la classe des fonctions de croissance polynomiale, appartenant \`a $C^0([0,T]\times ]0,\infty[) and C^{1,2}([0,T)\times]0,\infty[)$ donn\'ee par:
\begin{equation}
v(t,S)=E[e^{-r(T-t)}g(Se^{(r-\frac{1}{2}\sigma^2)(T-t)+\sigma (W_T -W_t)})]
\end{equation}
\end{thm}


\subsection{Formule de Black-Scholes}
\begin{thm}
	Le prix d'une option europeene (payoff)
\begin{equation}
g(S)=(S-K)_+
\end{equation}
dans le mod\`ele de B\&S est donn\'ee par
\begin{equation}
v(t,S)=C_{BS}(t,S)=S\mathcal{N}(d_1)-K e^{-r(T-t)}\mathcal{N}(d_2)
\end{equation}
o\`u
\begin{equation}
\mathcal{N}(x)=\int_{-\infty}^x \frac{e^{-\frac{y^2}{2}}}{\sqrt{2\pi}}dy
\end{equation}
\begin{equation}
d_1=\frac{\ln(\frac{S}{Ke^{-r(T-t)}})+\frac{1}{2}\sigma^2(T-t)}{\sigma \sqrt{T-t}}
\end{equation}
\begin{equation}
d_2=\frac{\ln(\frac{S}{Ke^{-r(T-t)}})-\frac{1}{2}\sigma^2(T-t)}{\sigma\sqrt{T-t}}=d_1-\sigma\sqrt{T-t}
\end{equation}

Le ratio de couverture (delta) est donn\'ee par
\begin{equation}
\Delta_{BS}(t,S)=\mathcal{N}(d_1)
\end{equation}

\end{thm}
\textbf{preuve}:

\begin{equation}
\begin{split}
C_{BS}(t,S)&=E[e^{-r(T-t)}(Se^{(r-\frac{1}{2}\sigma^2)(T-t)+\sigma(W_T-W_t)}-K)_+]\\
&=E[e^{-r(T-t)}(Se^{(r-\frac{1}{2}\sigma^2)(T-t)+\sigma(W_T-W_t)}-K) \mathds{1}_{\{Se^{(r-\frac{1}{2}\sigma^2)(T-t)+\sigma(W_T-W_t)}>K\}}]\\
&=P_1-P_2
\end{split}
\end{equation}

\begin{equation}
\begin{split}
P_2&=Ke^{-r(T-t)}E[\mathds{1}_{\{Se^{(r-\frac{1}{2}\sigma^2)(T-t)+\sigma(W_T-W_t)}>K\}}]\\
&=Ke^{-r(T-t)}P((r-\frac{1}{2}\sigma^2)(T-t)+\sigma(W_T-W_t)>ln(\frac{K}{S}))\\
&=Ke^{-r(T-t)}P(\frac{W_T-W_t}{\sqrt{T-t}}>\frac{ln(\frac{K}{S})-(r-\frac{1}{2}\sigma^2)(T-t)}{\sigma\sqrt{T-t}})\\
&=Ke^{-r(T-t)}P(\frac{W_T-W_t}{\sqrt{T-t}}>\frac{ln(\frac{Ke^{-r(T-t)}}{S})+\frac{1}{2}\sigma^2(T-t)}{\sigma\sqrt{T-t}})\\
&=Ke^{-r(T-t)}P(\frac{W_T-W_t}{\sqrt{T-t}}<d_2)\\
&=Ke^{-r(T-t)}\mathcal{N}(d_2)
\end{split}
\end{equation}

\begin{equation}
\begin{split}
P_1&=E[e^{-r(T-t)}Se^{(r-\frac{1}{2}\sigma^2)(T-t)+\sigma(W_T-W_t)}\mathds{1}_{\{Se^{(r-\frac{1}{2}\sigma^2)(T-t)+\sigma(W_T-W_t)}>K\}}]\\
&=E[Se^{-\frac{1}{2}\sigma^2(T-t)+\sigma(W_T-W_t)}\mathds{1}_{\{Se^{(r-\frac{1}{2}\sigma^2)(T-t)+\sigma(W_T-W_t)}>K\}}]
\end{split}
\end{equation}
Comme
\begin{equation}
W_T-W_t\sim\mathcal{N}(0,T-t)
\end{equation}
on a
\begin{equation}
W_T-W_t=\sqrt{T-t}Z
\end{equation}
avec $Z\sim \mathcal{N}(0,1)$.
\begin{equation}
\begin{split}
P_1&=\int_{\mathbb{R}}Se^{-\frac{1}{2}\sigma^2(T-t)+\sigma(W_T-W_t)}\mathds{1}_{\{Se^{(r-\frac{1}{2}\sigma^2)(T-t)+\sigma(W_T-W_t)}>K\}}\times \frac{e^{\frac{-z^2}{2}}}{\sqrt{2\pi}}dz\\
&=\int_{\{Se^{(r-\frac{1}{2}\sigma^2)(T-t)+\sigma(W_T-W_t)}>K\}} Se^{-\frac{1}{2}\sigma^2(T-t)+\sigma(W_T-W_t)}\frac{e^{\frac{-z^2}{2}}}{\sqrt{2\pi}}dz\\
&=\int_{\{\frac{1}{2}\sigma^2(T-t)+\sigma\sqrt{T-t}>ln(\frac{K}{S})\}} S\frac{e^{-\frac{(z-\sigma\sqrt{T-t})^2}{2}}}{\sqrt{2\pi}}dz
\end{split}
\end{equation}
En appliquant
\begin{equation}
	y=-\sigma\sqrt{T-t}
\end{equation}
on a
\begin{equation}
\begin{split}
P_1&=S\int_{\{\sigma\sqrt{T-t}(y+\sigma\sqrt{T-t})+(r-\frac{1}{2}\sigma^2)(T-t)>ln(\frac{K}{S})\}}\frac{e^{-y^2}}{\sqrt{2\pi}}dy\\
&=S\int_{\{y>\frac{ln(\frac{Ke^{-r(T-t)}}{S})-\frac{1}{2}\sigma^2(T-t)}{\sigma\sqrt{T-t}}\}}\frac{e^{-y^2}}{\sqrt{2\pi}}dy\\
&=S\mathcal{N}(d_1)
\end{split}
\end{equation}
{\color{red} Comment?}

\begin{equation}
\begin{split}
\frac{\partial C_{BS}}{\partial S}(t,S)&=\frac{\partial}{\partial S}E[e^{-r(T-t)}(Se^{(r-\frac{1}{2}\sigma^2)(T-t)+\sigma(W_T-W_t)}-K)_+]\\
&=E[e^{-r(T-t)}\mathds{1}_{\{Se^{(r-\frac{1}{2}\sigma^2)(T-t)+\sigma(W_T-W_t)}>K\}}]\\
&=\mathcal{N}(d_1)
\end{split}
\end{equation}

\subsubsection{Corrollaire}
Le prix d'un put europ\'een de payoff $g(S)=(K-S)_+$ dans le mod\`ele de B\&S est donn\'e par:
\begin{equation}
P_{BS}(t,S)=Ke^{-r(T-t)}\mathcal{N}(-d_2)-S\mathcal{N}(-d_1)
\end{equation}
Le ratio de couverture (delta du put) est donn\'e par:
\begin{equation}
\frac{\partial P_{BS}}{\partial S}(t,S)=\mathcal{N}(d_1)-1 
\end{equation}

Parit\'e Call-Put:
\begin{equation}
\begin{split}
&Call(t,S,K,T)-Put(t,S,K,T)\\
=& S-Ke^{-r(T-t)}\\
=&S-KB(t,T)
\end{split}
\end{equation}

\begin{equation}
\frac{\partial C_{BS}}{\partial S}(r,S)-\frac{\partial P_{BS}}{\partial S}(r,S) = 1-0=1
\end{equation}
\begin{equation}
\frac{\partial P_{BS}}{\partial S} = \frac{\partial C_{BS}}{\partial S}-1=\mathcal{N}(d_1)-1
\end{equation}

\begin{equation}
P_{BS}(t,S)=-S+Ke^{-r(T-t)}+S\mathcal{N}(d_1)-Ke^{-r(T-t)}\mathcal{N}(d_2)
\end{equation}
\begin{equation}
1-\mathcal{N}(x)=\mathcal{N}(-x)
\end{equation}
Donc 
\begin{equation}
P_{BS}(t,S)=Ke^{-r(T-t)}\mathcal{N}(-d_2)-S\mathcal{N}(-d_1)
\end{equation}
\begin{remq}
	\begin{equation}
\frac{\partial C_{BS}}{\partial S}=\mathcal{N}(d_1)>0
\end{equation}
et
\begin{equation}
\frac{\partial P_{BS}}{\partial S}=\mathcal{N}(d_1)-1<0
\end{equation}
\end{remq}
Ainsi, $C_{BS}$ est croissant et $P_{BS}$ est d\'ecroissant par rapport \`a $S$.

\subsection{Les grecques}

Pour comprendre le comportement des options en fonction des diff\'erents propri\'et\'es du mod\`ele, on calcule les subtilit\'es du prix Black Scholes  par rapport \`a ces propri\'et\'es:
\begin{enumerate}
\item Delta

Le delta est la sensibilit\'e du prix par rapport \`a la valeur actuelle de l'actif sous-jacent(underlying asset).
\begin{equation}
\Delta^c = \frac{\partial C_{BS}}{\partial S}=\mathcal{N}(d_1)
\end{equation}
\begin{equation}
\Delta^p=\frac{\partial P_{BS}}{\partial S}=\mathcal{N}(d_1)-1
\end{equation}

\item Gamma

Le gamma est d\'efini comme la d\'eriv\'ee du prix au b, en la d\'eriv\'ee du delta.
\begin{equation}
\Gamma^c=\Gamma^p=\frac{\partial^2 C_{B,S}}{\partial S^2}=\frac{\partial^2 P_{BS}}{\partial S^2}=\frac{n(d_1)}{S\sigma\sqrt{T-t}}
\end{equation}
\begin{equation}
n(x)=\frac{e^{-\frac{x^2}{2}}}{\sqrt{2\pi}}
\end{equation}

Puisque $\Gamma$ est positif, le prix du call et du put est convexe en $S$. Le Gamma est grand quand l'option est \`a la monnaie au proche de l'\'ech\'eance

\item le Vega

Le vega est la sensibilit\'e du  prix par rapport \`a la volatilit\'e:
\begin{equation}
\nu=\frac{\partial C_{BS}}{\partial \sigma} = \frac{\partial P_{BS}}{\partial\sigma}=Sn(d_1)\sqrt{T-t}
\end{equation}
Le prix du put et du call est croissant par rapport \`a la volatilit\'e. Le Vega est plus grande \`a la monnaie mais d\'ecroit pour les options proches de la maturit\'e.

\item Th\'eta

Le th\'eta est la sensibilit\'e par rapport au temps.
\begin{equation}
\Theta^c=\frac{\partial C_{BS}}{\partial t}=-\frac{Sn(d_1)\sigma}{2\sqrt{T-t}}-rKe^{-r(T-t)}\mathcal{N}(d_2)<0
\end{equation}
\begin{equation}
\Theta^p=\frac{\partial P_{BS}}{\partial t}=-\frac{Sn(d_1)\sigma}{2\sqrt{T-t}}+rKe^{-r(T-t)}\mathcal{N}(-d_2)
\end{equation}
Ainsi, Le prix du call est d\'ecroissant par rapport au temps

\item Rho

Le Rho est la sensibilit\'e par rapport au taux d'int\'er\^et $r$
\begin{equation}
\rho_{BS}^c=\frac{\partial C_{BS}}{\partial r}=K(T-t)e^{-r(T-t)}\mathcal{N}(d_2)>0
\end{equation}
\begin{equation}
\rho_{BS}^p=\frac{\partial P_{BS}}{\partial r}=-K(T-t)e^{-r(T-t)}\mathcal{N}(-d_2)<0
\end{equation}
Ainsi, call est croissant et le put est d\'ecroissant par rapport au taux d'int\'er\^et.

\end{enumerate}

\subsection{Erreur de couverture}

Dans le mod\`ele de B\&S, pour que l'option soit compl\`etement r\'epliqu\'ee, le portefeuille de couverture soit \^etre r\'eajust\'e en continue. En pratique, il est r\'eajust\'e \`a des dates discr\`etes, ce qui entra\^ine une erreur de couverture(erreur de discretisation).

Pour quantifier cette erreur, on \'ecrit la diff\'erence entre la valeur actualis\'ee d'une option et le portefeuille de couverture $V$ correspondant(r\'eajust\'e de mani\`ere discr\`ete \`a des dates $0=t_0<t_1<t_2<\ldots<t_n=T$)

\begin{equation}
\tilde{v}(t,\tilde{S})=e^{-RT}v(t,e^{RT}\tilde{S})
\end{equation}
\begin{equation}
\begin{split}
\tilde{v}(T,\tilde{S}_T)&=\tilde{v}(0,\tilde{S}_0)++\sum_{i=1}^n(\tilde{v}(t_i,\tilde{S}_{t_i})-\tilde{v}(t_{i-1},\tilde{S}_{t_{i-1}}))\\
&=\tilde{v}(0,\tilde{S}_0)+\sum_{i=1}^n \frac{\partial}{\partial t}[\tilde{v}(t_i, \tilde{S}_{t_i})]\Delta t_i,\frac{\partial}{\partial S}\tilde{v}(t_{i-1},\tilde{S}_{t_{i-1}})\Delta \tilde{S}_{t_i}+
\frac{1}{2}\frac{\partial^2}{\partial\tilde{S}^2}\tilde{v}(t_{i-1},\tilde{S}_{t_{i-1}}(\Delta\tilde{S}_{t_i})^2)
\end{equation}
Cherchons l'EDP v\'erifi\'ee par $\tilde{v}$:
\begin{equation}
\begin{split}
\frac{\partial\tilde{v}}{\partial t}&=-Re^{-RT}v(t,e^{RT}\tilde{S})+E^{-RT}\frac{\partial}{\partial t}v(t,e^{e^{RT}\tilde{S}})\\
&=-re^{-RT}v(t,e^{RT}\tilde{S})+e^{-RT}[\frac{\partial v(t,e^{RT}\tilde{S})}{\partial t}+Re^{RT}\tilde{S}\frac{\partial v(t,e^{RT}\tilde{S})}{\partial \tilde{S}}]
\end{split}
\end{equation}
On a
\begin{equation}
\frac{\partial v}{\partial t}=r(v-S\frac{\partial v}{\partial S})=\frac{1}{2}\sigma^2 S^2\frac{\partial^2v}{\partial S^2}
\end{equation}
Donc
\begin{equation}
\begin{split}
\frac{\partial\tilde{v}}{\partial t}=-Re^{RT}v(t,e^{RT}\tilde{S})+e^{-RT}[R(v(t,e^{RT}\tilde{S}-e^{RT}\tilde{S}\frac{\partial}{\partial S}v(t,e^{RT}\tilde{S})-\frac{1}{2}\sigma^2e^{2RT}(\tilde{S})^2\frac{\partial^2 v}{\partial \tilde{S}^2+Re^{-RT}\tilde{S}\frac{\partial v(t,e^{RT}\tilde{S})}{\partial S}))]\\
&=-\frac{1}{2}\sigma^2(\tilde{S})^2\frac{\partial^2\tilde{v}}{\partial\tilde{S}^2}
\end{split}
\end{equation}
Ainsi
\begin{equation}
\frac{\partial\tilde{v}}{\partial t}+\frac{1}{2}\sigma^2\tilde{S}^2\frac{\partial^2\tilde{v}}{\partial\tilde{S}^2}=0
\end{equation}

\begin{equation}
\begin{split}
\tilde{v}(T,\tilde{S}_T)\approx v(0,S_0)+\sum_{i=1}^n\{-\frac{1}{2}\sigma^2\tilde{S}_{t_i}^2\frac{\partial \tilde{v}}{\partial \tilde{S}}\Delta t_{i-1}+\frac{\partial\tilde{v}}{\partial\tilde{S}}\sigma\tilde{S}_{t_{i-1}}\Delta W_{t_i}+\frac{1}{2}\frac{\partial^2\tilde{v}}{\partial S^2}\sigma^2\tilde{S}_{t_{i-1}}(\Delta W_{t_i})^2\}\\
&=v(0,\tilde{S}_0)+\sum_{i=1}^n\sigma\tilde{S}_{t_{i-1}}\frac{\partial\tilde{v}}{\partial\tilde{S}}\Delta W_{t_i}+\sum_{i=1}^n\frac{1}{2}\sigma^2\tilde{S}_{t_{i-1}}\{(\Delta W_{t_i})^2-\Delta t_i\}
\end{split}
\end{equation}
\begin{equation}
\tilde{V}_t=\tilde{v}(0,\tilde{S}_0)+\sum_{i=1}^n(\tilde{V}_{t_i}-\tilde{V}_{t_{i-1}})
\end{equation}
On a 
\begin{equation}
d\tilde{V}_t=\frac{\partial v}{\partial S}d\tilde{S}_t
\end{equation}
\begin{equation}
\tilde{V}_T\approx \tilde{v}(0,\tilde{S}_0)+\sum_{i=1}^n\frac{\partial v}{\partial S}(t_{i-1},S_{t_{i-1}})\Delta\tilde{S}_{t_i}
\end{equation}
\begin{equation}
\frac{\partial}{\partial S}\tilde{v}(t,\tilde{S}_t)=\frac{\partial v}{\partial S}(t,S_t)
\end{equation}
\begin{equation}
\frac{\partial}{\partial S}[e^{-RT}v(t,R^{Rt}\tilde{S}_t)]=\frac{\partial}{\partial S}v(t,S_t)
\end{equation}
\begin{equation}
\tilde{V}_T-\tilde{v}(T,S_T)=-\frac{1}{2}\sum_{i=1}^n\sigma^2\tilde{S}_{t_{i-1}}^n\{\Delta W_{t_i}^2-\Delta t_i\}
\end{equation}
Aisni,
\begin{equation}
V_T-v(T,S_T)&=e^{RT}(\tilde{V}_T-\tilde{v}(T,\tilde{S}_T))\\
&=-\frac{e^{RT}}{2}\sum_{i=1}^n\sigma^2 \tilde{S}_{t_{i-1}}\{\Delta W_{t_i}^2-\Delta t_i\}
\end{equation}
\begin{equation}
E[(\Delta W_{t_i})^2-\Delta t_i]=0
\end{equation}
\begin{equation}
Var[(\Delta W_{t_i})^2-\Delta t_i]=2(\Delta t_i)^2
\end{equation}
\begin{equation}
V_T-v(T,S_T)=-\frac{e^{RT}\Delta t_i}{\sqrt{2}}\sigma^2\sum_{i=1}^n (e^{-Rt_{i-1}}S_{t_{i-1}})^2\frac{\partial^2 v}{\partial S^2}(\frac{\Delta W_{t_i}^2-\Delta t_i}{\sqrt{2}\Delta t_i})
\end{equation}
\begin{equation}
\begin{split}
Var[(\Delta W_{t_i})^2-\Delta t_i]&=Var[(\Delta W_{t_i})^2]\\
&=E[(\Delta W_{t_i})^4]-E[(\Delta W_{t_i})^2]^2\\
&=3(\Delta t_i)^2-(\Delta t_i)^2\\
&=2(\Delta t_i)^2
\end{split}
\end{equation}

\subsection{Robustesse de la formule de Black Scholes}
La formule de Black Scholes est souvent utilis\'ee dans le march\'e m\^eme pour les actifs dont les volatilit\'e n'est pas constante. Cette pratique est justif\'ee par la pratique appel\'ee robustesse de la formule de Black-Scholes.

On suppose 
\begin{equation}
dS_t=S_t (bdt+\sigma_t dW_t)
\end{equation}
sous $\mathbb{P}$, $(\sigma_t)_t$ est un processus stochastique. On va supposer que le vendeur couvre l'ioption en se servant de la formule de B\&S, c'est-\`a-dire, mettre en place un portefeuille de couverture qui coute
\begin{equation}
\Delta=\frac{\partial C_{BS}(t,S_t)}{\partial S}
\end{equation}
de l'actif risqu\'e. La formule de $C_{BS}$ est calcul\'e avec une valeur constante $\Sigma$.

\begin{equation}
d\tilde{V}_T=\Delta_td\tilde{S}_t
\end{equation}
\begin{equation}
\tilde{V}_T=\tilde{V}_0-\int_0^T\Delta_td\tilde{S}_t
\end{equation}
\begin{equation}
(S_T-K)_+=C_{BS}(T,S_T)
\end{equation}
\begin{equation}
e^{-RT}(S_T-K)_+=\tilde{C}_{BS}(T,\tilde{S}_T)(\tilde{C}_{BS}(t,S)-e^^{-RT}C_{BS}(t,e^{RT}S))
\end{equation}
\begin{equation}
\tilde{C}_{BS}(T,\tilde{S}_T)=\tilde{C}_{BS}(0,\tilde{S}_0)+\int_0^T\frac{\partial \tilde{C}_{BS}(t,S_t)}{\partial t}dt+\int_0^T\frac{\partial\tilde{C}_{BS}}{\partial S}(t,\tilde{S}_t)d\tilde{S}_t+\frac{1}{2}\int_0^T\frac{\partial^2 \tilde{C}_{BS}}{\partial S^2}(t,\tilde{S}_t)\sigma_t^2S_t^2dt
\end{equation}
\begin{equation}
\tilde{V}_T-\tilde{H}_T=-\int_0^T\frac{\partial \tilde{C}_{BS}}{\partial t}(t,\tilde{S}_t)dt-\frac{1}{2}\int_0^T\sigma_t^2\tilde{S}_t^2\frac{\partial^2\tilde{C}_{BS}}{\partial S^2}(t,\tilde{S})dt
\end{equation}
\begin{equation}
\frac{\partial\tilde{C}_{BS}}{\partial t}+\frac{1}{2}\Sigma_t^2\tilde{S}_t^2\frac{\partial^2\tilde{C}_{BS}}{\partial S}=0
\end{equation}
Ainsi,
\begin{equation}
\tilde{V}_T-\tilde{H}_T=\int_0^T\frac{1}{2}(\Sigma^2-\sigma_t^2)\tilde{S}_t^2\frac{\partial^2\tilde{C}_{BS}}{\partial S^2}(t,\tilde{S}_t)dt
\end{equation}
Donc,
\begin{equation}
V_T-H_T=e^{RT}\int_0^T\frac{1}{2}(\Sigma^2-\sigma_t^2)e^{-2RT}S_t^2\frac{\partial^2\tilde{C}_{BS}}{\partial S^2}(t,\tilde{S}_t)dt
\end{equation}
\begin{equation}
\frac{\partial^2\tilde{C}_{BS}(t,\tilde{S}_t)}{\partial S^2}=e^{RT}\frac{\partial^2C_{BS}(t,S_t)}{\partial S^2}
\end{equation}
Ainsi,
\begin{equation}
V_T-H_T=\int_0^T\frac{1}{2}e^{R(T-t)}S_t^2\frac{\partial^2C_{BS}(t,S_t)}{\partial S^2}(\Sigma^2-\sigma^2_t)dt
\end{equation}
Si $\Sigma\geq\sigma_t$, alors $V_T\geq H_T$

\subsection{R\'einterpr\'etation avec le point de vue probabilit\'e risque neutre}

\begin{equation}
dS_t=S_t(bdt+\sigma dW_t)
\end{equation}
sous $\mathbb{R}$
\begin{equation}
\begin{split}
d((e^{-RT}S_T)&=-Re^{-RT}S_tdt+e^{-RT}dS_t\\
&=-Re^{-RT}S_tdt+e^{-RT}S_t(bdt+\sigma dW_t)\\
&=e^{-RT}S_t(b-R)dt+\sigma e^{-RT}S_t+d W_t
\end{split}
\end{equation}
\begin{equation}
\begin{split}
d(\tilde{S}_t)&=\tilde{S}_t(\sigma dW_t+(b-R)dt)\\
&=\tilde{S}_t\sigma (dW_t+\frac{b-R}{\sigma}dt)
\end{split}
\end{equation}
\begin{equation}
\tilde{S}_T=\tilde{S}_t+\int_t^T\sigma\tilde{S}_u(dW_u+\frac{b-R}{\sigma}du)
\end{equation}
\begin{equation}
E^Q[e^{-RT}S_T|S_t]=e^{-RT}S_t
\end{equation}
\begin{equation}
\frac{dQ}{dP}=e^{-\frac{1}{2}(\frac{b-R}{\sigma})^2T-(\frac{b--R}{\sigma})W_T}
\end{equation}
implique que 
\begin{equation}
\tilde{W}_t=W_t+\frac{b-R}{\sigma}t
\end{equation}
 est un mouvement brownien standard sous $Q$.
 
 Sous $Q$,
\begin{equation}
\tilde{S}_T=\tilde{S}_t+\int_t^T\sigma\tilde{S}_ud\tilde{W}_u
\end{equation}
\begin{equation}
C_{BS}(0,S_0)=E^Q[e^{-RT}(S_T-K)_+]
\end{equation}
\begin{equation}
d\tilde{S}_t=\tilde{S}_t+dW_t
\end{equation}
\begin{equation}
\begin{split}
d(S_t)&=d(e^{RT}\tilde{S}_t)\\
&=Re^{RT}\tilde{S}_tdt+e^{RT}d\tilde{S}_t\\
&=RS_tdt+e^{RT}\sigma\tilde{S}_td\tilde{W}_u\\
&=S_t(Rdt+\sigmad\tilde{W}_u)
\end{split}
\end{equation}
Donc,
\begin{equation}
S_t=S_0e^{(R-\frac{1}{2}\sigma^2)T+\sigma\tilde{W}_T}
\end{equation}

\subsection{Volatilit\'e implicite}
Vega:
\begin{equation}
\nu=\frac{\partial C_{BS}}{\partial \sigma}(\sigma)=Sn(d_1)\sqrt{T}>0
\end{equation}
\begin{equation}
C_{BS}(0)=(S-Ke^{-RT})_+
\end{equation}
\begin{equation}
\lim_{\sigma\to\infty}C_{BS}(\sigma)=S
\end{equation}

Smile de volatilit\'e :)

Avec un prix de l'option sur le march\'e, on r\'esoudre $C_{BS}(\sigma,K)=P$ en changant $K$, et on obtiendra une courbe $\sigma_{implicit}\~K$, ce qui est une courbe de smile.

\section{Examen}
5 exercises

1. Question de cours

les graphs d\'esinn\'ees

Les autres exercises seront bas\'ees sur les exercises que l'on a faites.

Valorisation dans le cadre discr\`ete

5e exercise concerne le chap\^itre de Black-Scholes

\end{document}
