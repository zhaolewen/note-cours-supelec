\documentclass{article}
\linespread{1.5}

\usepackage{amsmath}
\usepackage{amsfonts}
\usepackage{mathrsfs}
\usepackage{dsfont}

\title{Mineur Finance 1}

\begin{document}
\maketitle

\tableofcontents
\pagebreak

\section{Cadre de valorisation des produite devive}
\subsection{Introduction}
\subsection{Hypothese}
\subsection{Propriete}
\subsection{Options americaines}
Option am\'ericaine similaire aux options europeens mais l'exercise du droit pourra se faire \`a toute date $t$ avant maturit\'e $T$.

On note $CallAmer_t(T, K)$ et $PutAmer_t(T, K)$ les prix des calls et puts am\'ericains au $t$.

On a 
\begin{equation}
CallAmer_t(T,K) >= Call_t(T, K)
\end{equation}

\begin{equation}
PutAmer_t(T, K) >= Put_t(T, K)
\end{equation}

Propri\'et\'e:
En l'absence de divident, on a:

\begin{equation}
CallAmer_t(T, K) = Call_t(T, K)
\end{equation}

Preuve:

\begin{equation}
Call_t(T, K) >= (S_t- K)_+
\end{equation}

Il n'est pas optimal d'exercer avant la maturit\'e. Donc $CallAmer_t(T, K) = Call_t(T, K)$

Remarque: Pour le Put, la propri\'et\'e n'est pas vraie.

\begin{equation}
Put_t(T, K) <= KB(t, T)
\end{equation}

Pour un certain scenario, le prix de put est inf、‘erieur au prix d'exercise. $Put_t(T, K)<(S_t-K)_+$.

Pour $S_t$ tel que $KB_t(T)< K-S_t <=> S_t < (1-B_t(T))K$
Donc, on a
\begin{equation}
Put_t(T, K)< K-S_t
\end{equation} 

\section{Th\'eorie de la valorisation dans le cas discr\`et \`a une p\'eriode}

\subsection{Introduction}
Nous avons vu que les contrats \`a termes peuvent \^{e}tre valoris\'es par le principe de non arbitrage alors que ce principe ne donne que des bornes sur les prix des options. La valorisation de ces produits assez complexes n\'ecessite l'introduction des modeles probabilistes pour d\'ecrire les scenarios possibles du march\'e.

Dans ce marche, si on trouve une strategie qui r\'eplique le payoff de l'option, alors la portefeuille construite par cette strategie r\'eplique l'option.

Par l'absence de opportunite d'arbitrage, la valeur de ce portefeuille est la m\^{e}me que celui de l'option. Cependant, ce prix est d\'ependant du mod\`ele qui doit etre proche de la r\'ealit\'e.

\subsection{Mod\`ele \`a une p\'eriode}

On consir\`ere qu'il y a $d+1$ actifs financiers et $2$ dates,: $t=0, t=1$. L'al\'ea du mod\`ele est represent\'e par $K$ \'etats dans $\Omega=\{\omega_1,\ldots,\omega_K\}$.

En $t=1$, avec des probabilit\'es $ \{p_1,\ldots,p_K\},(p_i>0) $.

Le prix de l'actif $i$ au $t$ est not\'e $S_t^i, 0\leq i\leq d$. $S^0$ represente l'actif saus risque $(S_0^0=1, S_1^0=e^R\approx 1+R)$
Une strategie $\theta\in\mathbb{R}^{d+1}$ est le vecteur contenant le nombre d'unit\'e de chaque actif dans ce portefeuille $V$ entre $t=0$ et $t=1$. 

La valeur de $V$ en $t$
\begin{equation}
V_t=V_t^\theta=\sum_{i=0}^d\theta_i S_t^i
\end{equation}
Le gain est defini par 
\begin{equation}
G=V_1-V_0=\sum_{i=0}^d \theta_i (S_1^i-S_0^i)=\sum_{i=0}^d \theta_i \Delta S^i
\end{equation}

On note $\tilde{S}_t^i=\frac{S_t^i}{S_t^0}$, $\tilde{V}_t=\frac{V_t}{S_t^0}$ et $\tilde{G}_t=\frac{G}{S_t^0}$,
alors
\begin{equation}
\left \{ \begin{array}{rcl}
&\tilde{V}_t =\theta_0+\sum_{i=0}^{d}\theta_i \tilde{S}_t^i\\
&\tilde{G} =\tilde{V}_1-\tilde{V}_0=\sum_{i=1}^d \theta_i (\tilde{S}_1^i-\tilde{S}_0^i)=\sum_{i=1}^d\theta_i\Delta \tilde{S}^i
\end{array}\right.
\end{equation}

\subsection{R\'esultats de non-arbitrage}
\subsubsection{D\'efinition}

Une opportunite d'arbitrage est une strategie $\theta$ tel que 
\begin{enumerate}
	\item  $V_0^\theta =0$
	\item $P(V_1^\theta >=0) = 1$ et $P(V_1^\theta>0)>0$ 
\end{enumerate}
o\`u
\begin{enumerate}
	\item $P(G>=0) = 1$
	\item $P(G>0) > 0$
\end{enumerate}

Commencer avec une \'etat avec valeur $0$, et la probabilite de gagner de l'argent est superierue \`a $0$.

\subsubsection{D\'efinition 2}

Une mesure de probabilit\'e $Q$ est dite risque neutre si:
\begin{enumerate}
	\item $Q\sim P$, c'est a dire, $Q(\omega_i)>0, \forall 1<=i<=K$
	\item $E^Q[\tilde{S}_1^i]=\tilde{S}_0^i, \forall i$
\end{enumerate}

\subsubsection{Th\'eor\`eme}
Le march\'e n'admet pas d'opportunit\'e d'arbitrage si et seulement s'il existe au moins une mesure de probabilit\'e risque neutre.

Preuve:

($=>$) Supposons qu'il existe une mesure de probabilit\'e risque neutre. Supposons qu'il existe une opportunit\'e d'arbitrage, donc 
\begin{equation}
Q(G^\theta>=0)=1, Q(G^\theta>0)>0
\end{equation}

\begin{equation}
E^Q[\tilde{G}^\theta]=0
\end{equation}
contradiction.

($<=$) Supposons qu'il y a absence d'opportunit\'e d'arbitrage
\begin{equation}
\mathcal{E}=\{(q_1,\ldots,q_K)|\sum_{i=1}^K q_i, q_i>0\}
\end{equation}
\begin{equation}
C=\{E^Q[\Delta\tilde{S}]|Q\in\mathcal{E}\}, \Delta \tilde{S}=(\Delta \tilde{S}^1,\ldots, \Delta\tilde{S}^d)^T
\end{equation}
Il faut d\'emontrer que $0\in C$.

Supposons $0\notin C$, puisque $C$ est convexe, d'apr\`es le theoreme de separation des convexes, alors 
\begin{equation}
\exists \alpha \in \mathbb{R}^d, \alpha^t >= 0, \forall x\in C
\end{equation}
\begin{equation}
\exists x_0\in C, \alpha^t x_0 > 0
\end{equation}
\begin{equation}
x_0=E^{Q_0}[\Delta\tilde{S}]
\end{equation}
Soit $\alpha_0$ arbitraire, on consid\`ere la strategie $\theta=(\alpha_0, \alpha)$
On a $\forall \theta\in\mathcal{E}, Q(\tilde{G}^\theta>=0)=1, Q_0(\tilde{G}^\theta>0)>0$
On a $\forall Q\in\mathcal{E}, E^Q[\tilde{G}^Q]>=0, E^Q[\tilde{G}^Q]>0$

S'il existe $\omega_i$ tel que $G^Q(\omega_i)<0$, on va construire une mesure $Q^t$:
\begin{equation}
\begin{split}
	Q^\epsilon(\omega)= & 1-\frac{K-1}{K}\epsilon , \omega = \omega_i \\
	\frac{\epsilon}{K} & \omega\neq \omega_i
\end{split}
\end{equation}
on peut choisir $\epsilon$ suffisamment actif tel que $E^{Q_\epsilon}[\tilde{G}^{\theta}]<0$. Donc $\forall i, G^\theta(\omega_i)>=0 => Q^0(G^\theta>=0)=1$ mais puisque $E^{Q_0}[\tilde{G}^\theta]>0$, alors $Q^0(\tilde{G}^theta)>0$

$=>$ Il y a arbitrage. Contradition, donc $0\in C => \exists $ une mesure de proba risque neutre.

\textbf{Exemple}:
\begin{equation}
\Omega=\{\omega_1, \omega_2,\omega_3, \omega_4\}, \mathbb{P}=\{0.3, 0.3,0.2,0.2\}
\end{equation}
$R=0.25$, $S^0$: Actif sans risque vaux $1$ en $t=0$.
$S^1$: une action qui vaut $1$ en $t=0$.
$S^2$: un call sur $S^1$ de strike $K=1$ qui vaut $0.3$ en $t=0$.
\begin{center}
\begin{tabular}{l||l|l|l|l|l}
		& $S_0^i$ & $w_1$ & $w_2$ & $w_3$ & $w_4$\\ \hline \hline
	$i=0$ & 1 & 1.25 & 1.25 & 1.25 & 1.25 \\	 \hline 
	$i=1$ & 1 & 0.5 & 1 & 1.5 & 2 \\ \hline 
	$i=2$ & 0.3 & 0 & 0 & 0.5 & 1 \\ 
\end{tabular}
\end{center}

Un actif vaut toujours la m\^{e}me valeur peu importe l'\'etat.

Exemple:
\begin{enumerate}
	\item Avec la strategie $\theta=(0.7, -1, 1)$, calculer la gain $G$ de $V^t$
	\item Est-ce qu'il y a une opportunit\'e d'arbitrage dans ce march\'e?
\end{enumerate}

\begin{equation}
G^\theta(\omega_i)=V_1(\omega_i)-V_0, V_0=1*0.7-1+0.3=0
\end{equation}

\begin{equation}
G^\theta(\omega_1)=0.375, G^\theta(\omega_{2,3,4})=-0.125
\end{equation}

Cherchons $Q=\{q_1,q_2,q_3,q_4\}$ tel que $0<q_i$ et
\begin{equation}
\left\{ \begin{array}{rcl}
	q_1+q_2+q_3+q_4 &= 1 \\
	\frac{0.5}{1.25} q_1+\frac{1}{1.25} q_2+\frac{1}{1.25} q_3 + \frac{2}{1.25} q_4 &= 1 \\
	\frac{0.5}{1.25} q_3 + \frac{1}{1.25} q_4 &=0.3
\end{array}\right.
\end{equation}

Et on a
\begin{equation}
\left\{ \begin{array}{rcl}
q_1&=\frac{1}{4} \\
q_2&=q_4 \\ 
q_3&=\frac{3}{4}-2q_4
\end{array}\right.
\end{equation}

l'ensemble des mesures risque neutre est :
\begin{equation}
\{(\frac{1}{4},\lambda,\frac{3}{4}-2\lambda)|0<\lambda<\frac{3}{8}\}=empty => AOA
\end{equation}

\section{March\'e complet et valorisation d'actif conditionnel}

On suppose que le march\'e est sans arbitrage.

\begin{enumerate}
	\item Def 1
	
Un actif conditionnel est une variable al\'eatoire $H$ dans $(\Omega, F, P)$.
	
	\item Def 2
	
	Un actif conditionnel est r\'ealisable s'il existe une strat\'egie $\theta$ tel que $H=V_1^\theta$
	
	\item Remarque
	
	Par AOA, on a prix de l'actif conditionnel est $P(H)=V_0^\theta$.
\end{enumerate}

Proposition: Le juste prix d'un actif conditionnel r\'ealisable $H$ est:
\begin{equation}
P(H) = E^Q[\tilde{H}]=E^Q[\frac{H}{S_1^\theta}]=E^Q[\frac{H}{1+R}]
\end{equation}
o\`u $Q$ est une mesure de proba risque neutre.

\textbf{preuve}: 
$\theta$ est une strat\'egie tel que $V_1^\theta=H$
\begin{equation}
\begin{split}
E^Q[\tilde{H}]&=E^Q[\tilde{V}_1^\theta]\\
&=E^Q[\tilde{V}_0^\theta+\sum_{i=1}^d \theta_i \Delta \tilde{S}^i]\\
&=V_0^\theta+\sum_{i=1}^d \theta_i E^Q[\Delta \tilde{S}^i]\\
&=V_0^\theta=P(H)
\end{split}
\end{equation}


\subsubsection{Def 3}

Un march\'e est complet si tout actif conditionnel est r\'ealisable.


Remarque: 
$A$: matrice des payoff: $A_{(i,k)}(S_1^i(w_k))$. Un actif conditionnel est r\'ealisable si $\exists \theta$ tel que $\theta^T A=H$. Et donc le march\'e est complet si $\theta^T A=H$ a une solution pour tout $H\in \mathbb{R}^K$.	

Donc il faut que $A$ a au moins $K$ lignes lin\'eairement ind\'edendantes.[une condition n\'ecessaire est donc $d+1>=K$]

\textbf{Th\'eor\`eme}: Un march\'e sans arbitrage est complet ssi il existe une seule mesure de probabilit\'e risque neutre.

Preuve: Supposons que le march\'e est complet. Prenons l'actif conditionnel $\mathbb{1}_{\{w_k\}}(1<=k<=K)$. Donc pour toute mesure de proba risque neutre $q$ on a:
\begin{equation}
P_k=P(\mathbb{1}_{\{w_k\}})=E^Q[\frac{\mathbb{1}_{\{w_k\}}}{1+R}]=\frac{1}{1+R}Q(\{w_k\})
\end{equation}
=> Unicit\'e de $Q$.

Supposons que la mesure risque neutre est unique. D\'emontrer que le march\'e est complet.

Si le march\'e n'est pas complet 
\begin{equation}
	\exists z\in\mathbb{R}^{d+1}, z\neq 0
\end{equation}
tel que $z^tA0$. Soit
\begin{equation}
	Q_i|Q_2=\frac{Q(w_k)+\epsilon z_k}{1+\epsilon \sum z_k}
\end{equation}. 
Il suffit de prendre $\epsilon$ suffisamment petit, $0<Q_2(w_k)<1$.
\begin{equation}
	\begin{split}
		E^{Q_\epsilon}&=\frac{\sum_{k+1}^K (Q(w_k)+\epsilon z_k \tilde{S}^i(w_k)}{1+\epsilon \sum z_k}\\
		&=\sum_{k=1}^K Q(w_k) \tilde{S}_1^i(w_k)\times\epsilon \sum_k \mathds{1}^{K} z_k \tilde{S}_1^i (w_k)\\
		&=E^Q[\tilde{S}_1^i]\\ 
		&=\tilde{S}_0^i
	\end{split}
\end{equation}

\textbf{Exercise}
1. $\Omega=\{w_1, w_2\}$, on cherche le modele suivant:

$S_0^0=1 -> 1+\Omega(1=10\%)$

$S_0^10->12, w_1,P_1$
$-> 11, w_2,P_2$

Est-ce que le march\'e est sans abritrage? Si ce n'est pas le cas, mettre au cas une strategie d'arbitrage.

Cherchons les mesures proba risque neutre
\begin{center}
\begin{tabular}{l|lll}
   & $S_0^1$ & $w_1$ & $w_2$ \\ \hline 
 $i=0$ & 1 & 1.1 & 1.1 \\
 $i=1$ & 10 & 12 & 11 \\
\end{tabular}	
\end{center}

Cherchons $P_1, P_2$ tel que 
\begin{equation}
P_1>0, P_2>0, E[\tilde{S_1}^0] = \tilde{S}_0^0, E[\tilde{S}_1^1]=\tilde{S}_0^1
\end{equation}
\begin{equation}
P_1*\frac{1.1}{1.1}+P_2\frac{1.1}{1.1}=\frac{1}{1} and P_1\frac{12}{1.1}+P_2\frac{11}{1.1}=\frac{10}{1}
\end{equation}
=>
\begin{equation}
P_1=0, P_2=1
\end{equation}
 
$P_1>0$ n'est pas v\'erifi\'e, donc il existe opportunit\'e d'arbitrage.
On veut 10 de $S^0$ et on ach\`ete 1 de $S^1$,
$V_0=10S_0^0+S_0^1=0$

Donc 
\begin{equation}
V_1-> -10(1+R)+12=1
\end{equation}
\begin{equation}
->-10(1+R)+11=0
\end{equation}


2. $\Omega=\{w_1, w_2\}$, on cherche le modele suivant:

$S_0^0=1 -> 1+\Omega(1=10\%)$

$S_0^10->12, w_1, P_1$
$-> 8, w_2, P_2$

\begin{enumerate}
\item Est-ce que le marche est sans arbitrage(complet)?
\item Calculer le prix d'un call de strike 10
\item Calculer de deux fa\c{c}ons le prix d'un put de strike 10
\end{enumerate}

Solution:$P_1=\frac{3}{4}, P_2=\frac{1}{4}$, donc il existe unique mesure probabilite risque neutre. Donc le march\'e est sans arbitrage(complet).

2) Puisque le march\'e est complet et sans arbitrage, alors:
\begin{equation}
Call_0(1,10) = E^Q[\frac{(S_1^1-10)_+}{1+R}]=\frac{\frac{3}{4}*2+\frac{1}{4}*0}{1.1}=\frac{15}{11}
\end{equation}

3) 
\begin{equation}
c(1,10)-P(1,10)=S_0^1-\frac{K}{1+R}=\frac{10}{11}
\end{equation}
\begin{equation}
=>P(1,10)=\frac{5}{11}
\end{equation}

\begin{equation}
P(1,10) = E^Q[\frac{(10-S_1^1)_+}{1+R}]=0\times\frac{3}{4}+\frac{2}{1.1}\times\frac{1}{4}=\frac{5}{11}
\end{equation}

\section{Le mod\`ele binominale \`a une p\'eriode}

$T=1,\Omega=\{w_u,w_d\},F=P(\Omega)$,$P$ mesure de proba dans $(\Omega,F),(0<P(w_u)<1)$

\begin{equation}
S_0^0=1->1+R
\end{equation}
\begin{equation}
S_0^1=s->us, w_u,P_1(u>=d)
\end{equation}
\begin{equation}
->ds, w_d, P_2
\end{equation}

\subsubsection{Proposition} 
\begin{enumerate}
\item Le mod\`ele est viable (sans arbitrage), ssi
\begin{equation}
d<1+R<u
\end{equation}
(condition de non-arbitrage)

\item Le prix d'un actif conditionnel $H$ est:
\begin{equation}
P(H)=E^Q[\frac{H}{1+R}]
\end{equation} 
o\`u
\begin{equation}
Q(w) = \left\{\begin{array}{rcl}
	\frac{1+R-d}{u-d} & si w=w_d\\
\frac{u-1-R}{u-d} & si w_u
\end{array} \right.
\end{equation}
En plus, ;a portefeuille de couverture est donn\'ee par $(\theta_0,\theta_1)$ o\`u 
\begin{equation}
\theta_1=\frac{H(w_u)-H(w_d)}{us-ds}
\end{equation}
et
\begin{equation}
\theta_0=P(H)-\theta_1 s
\end{equation}
\end{enumerate}

\textbf{Preuve}
\begin{enumerate}
\item Cherchons des mesures de proba risque neutre $Q=\{q_u,q_d\}(q_u=Q(w_u),q_d=Q(w_d))$
\begin{equation}
\left\{ \begin{array}{rcl}
q_u+q_d&=1\\
\frac{q_uus+q_d ds}{1+R}&=s
\end{array}\right. <=>
\left\{\begin{array}{rcl}
q_d&=1-q_u\\
q_us(u-d)&=s(1+R)-ds
\end{array}\right.
\end{equation}
$<=>$
\begin{equation}
q_d=\frac{u-1-R}{u-d}\\
q_u=\frac{1+R-d}{u-d}
\end{equation}

Donc $\exists !$ mesure de proba risque neutre ssi $d<1+R<u$

\item Le march\'e est sous arbitrage et complet

\begin{equation}
=>P(H)=E^Q[\frac{H}{1+R}]
\end{equation}
Cherchons $(\theta_0,\theta_1)$ tel que $\theta_0S_1^0 + \theta_1S_1^1=H$
\begin{equation}
\left\{\begin{array}{rcl}
\theta_0(1+R)+\theta_1(us)&=H(w_u)\\
\theta_0(1+R)+\theta_1(ds)&=H(w_d)
\end{array}\right.
\end{equation}
\begin{equation}
=>\theta_1(us-ds)=H(w_u)-H(w_d)
\end{equation}
\begin{equation}
=>\theta_1=\frac{H(w_u)-H(w_d)}{us-ds}
\end{equation}
\begin{equation}
\theta_0\times 1+\theta_1 s=P(H)
\end{equation}
\begin{equation}
=>\theta_0=P(H)-\theta_1 s
\end{equation}
La strategie de couverture est not\'e $(\theta_0,\theta_1)$ ou $(P(H),\theta_1)$
$=>$ le prix \`a definir pour l'acheter si le vendeur veut eliminer tout risque.
\end{enumerate}

\section{Th\'eor\`em de Valorisation dans le cas discr\`ete multi-p\'eriode}
\subsection{Mod\`ele}
On consid\`ere qu'il y a $n$ p\'eriodes et $n+1$ dates ${t_0=0,t_1,\ldots,t_n=T}$. L'al'ea est repr\'esent\'e par un espace de probabilit\'e $(\Omega,\mathcal{F},P)$. Pour sp\'ecifier l'information disponible en $t$, on introduit la filtration $\mathbb{F}=\{\mathcal{F}_0,\ldots,\mathcal{F}_n\}$. 

On suppose $\mathcal{F}_0=\{\empty,\Omega\},\mathcal{F}_n=\mathcal{F}=P(\Omega)$. LEs actifsrisqu\'e sont mod\'elis\'es par $d$ processus stocastiques $(S^i_j)_{j=0,\ldots,n}^{i=1,\ldots,d}$ comme $\mathbb{F}$-adapt\'e. L'actif sans risque est d\'efini $(S_j^0)_{j=0,\ldots,n}$
\begin{equation}
S_0^0=1,S_j^0=(0+R)S_{j-1}^0
\end{equation} 
$R_j$ est le taux d'int\'er\^et pour la p\'eriode $[t_{j-1},t^j]$.

\subsubsection{Def 1}
Une strat\'egie d'investissement admissible est un processus stocastique $(\theta_j^i)_{j=0,\ldots,n}^{i=0,\ldots,d}$ o\`u $\theta_j^i$ est la quantit\'e investie dans l'actif $i$ entre $t_{j-1}$ et $t_j$ qui doit \^etre connu en $t_{j-1}.($\theta$ est pr\'evisible).

\textbf{Remarque}:
La valeur d'un portefeuille peut varier par deux possibilit\'es

$->$ possibilit\'e 1: variation des prix d'actifs
$->$ possibilit\'e 2: versement/retrait d'argent

On va introduire la possibilit\'e 2 aux strat\'egies. 

\subsubsection{Def 2}
Une strat\'egie $\theta$ est dite audofinanc\'ee si 
\begin{equation}
\sum_{i=0}^d\theta_j^i S_j^i=\sum_{i=0}^d\theta_{j-1}^i S_j^i, \forall j=1,\ldots,n-1
\end{equation}

\subsubsection{Def 3}
Le processus de gain d'une strat\'egie est un processus stocastique qui d\'ecrit le changement de valeur duportefeuille associ\'e suite aux variation des prix d'actif. 

On va le noter $G$
\begin{equation}
G_0=0,G_t=\sum_{j=1}^t\sum_{i=0}^d \theta_j^i\Delta S_j^i,\Delta_j^i=S_j^i-S_{j-1}^i
\end{equation}
pour une strat\'egie auto-financ\'ee,
\begin{equation}
V_t=V_0+G_t
\end{equation}
Comme dans le cas d'une p\'eriode, on note:
\begin{equation}
\tilde{S}_j^i=\frac{S_t^i}{S_t^0},\tilde{V}_t=\frac{V_t}{S_t^0}=\tilde{V}_0+\tilde{G}_t
\end{equation}
\begin{equation}
\tilde{G}_t=\sum_{j=1}^t\sum_{i=1}^d \theta_j^i \Delta \tilde{S}_j^i
\end{equation}

\subsection{R\'esultats de march\'e viable et complet}

\subsubsection{Def 1}
Une opportunit\'e d'arbitrage est une strat\'egie audo-financ\'ee $\theta$ tel que:
\begin{equation}
\left\{\begin{array}{rcl}
V_0^\theta=0 \\
P(V_n^\theta >= 0)=1 et P(V_n^\theta >0)>0
\end{array}\right.
\end{equation}
ou
\begin{equation}
\left\{\begin{array}{rcl}
P(\tilde{G}_n^\theta>=0)=1\\
P(\tilde{G}_n^\theta>0)>0
\end{array}\right.
\end{equation}

\subsubsection{Th\'eor\`eme}
Un march\'e financ\'e \`a $n$ p\'eriode est viable ssi il existe une mesure de probabilit\'e risque neutre $Q$. C'est-\`a-dire,
\begin{equation}
\left\{\begin{array}{rcl}
Q(w)>0,\forall w\in\Omega\\
E^Q[\tilde{S_{t+1}^i}|\mathcal{F}_t]=\tilde{S}_t^i,\forall 0<=i<=d,0<=t<=n-1
\end{array}\right.
\end{equation}
(Les prix actualis\'es sont de $Q$-martingale).

\subsubsection{Proposition}
Le juste prix d'un actif conditionnel r\'ealisable $H$ \`a la date $t$ est(l'actif verse $H$ en $T$):
\begin{equation}
P_t(H)=S_t^0 E^Q[\frac{H}{S^0_t}|\mathcal{F}_t]
\end{equation}
($Q$ est la mesure risque neutre)

\textbf{Preuve}:
Il existe une strat\'egie \theta tel que 
\begin{equation}
H=\sum_{i=0}^d \theta_i S_t^i
\end{equation}
Donc
\begin{equation}
\begin{split}
S_tE^Q[\frac{H}{S_n^0}|\mathcal{F}_t]&=S_tE^Q[\frac{\sum\theta_i S_n^i}{S_n^0}|\mathcal{F}_t]\\
&=S_t\sum_{i=0}^d\theta_i E^Q[\tilde{S}_n^i|\mathcal{F}_t]\\
&=S_t\sum_{i=0}^d\theta_i\tilde{S}_t^i\\
&=\sum_{i=0}^d\theta_i S_t^i=V_t=P_t(H)
\end{split}
\end{equation}

\subsection{Th\'eor\`eme}
Un march\'e viable est complet ssi la moyenne de probabilit\'e risque neutre est unique.

\subsection{Mod\`ele binomiale multi-p\'eriode(Mod\`ele de Lex-Ross-Robinstein)}
\begin{equation}
\Omega=\{-1,1\}^n,\mathcal{F}=P(\Omega)
\end{equation}
Soit $(z_k)_{k>=0}$ une suite de variable al\'eatoires ind\'ependantes tel que $P(z_k=1)=P(z_k=-1)=\frac{1}{2}$
\begin{equation}
\mathcal{F}_0=\{\empty,\Omega\},\mathcal{F}_k=\sigma(z_0,\ldots,z_k),\mathbb{F}=\{\mathcal{F}_0,\ldots,\mathcal{F}_n\}
\end{equation}

$T>0, t_i=\frac{iT}{n}$: diviser $[0,T]$ en $n$ parties.

On d\'efinit:
\begin{equation}
\left\{\begin{array}{rcl}
b_n=b\frac{T}{n}\\
\sigma_n=\sigma\frac{T}{n}
\end{array}\right.
,b,\sigma>0
\end{equation}

\begin{equation}
S_0^0=1, S_j^0=e^{\Omega\frac{T}{n}}S_{j-1}^0,(R=e^{\Omega\frac{T}{n}}-1\tilde \frac{tT}{n})=(1+R)S_{j-1}^0
\end{equation}
\begin{equation}
\begin{split}
S_0^1=s,S_j^1&=se^{jb_n+\sigma_n\sum{k=1}^j z_k}\\
&=S_{j-1}^1 e^{b_n+\sigma_n z_j}, j=1,\ldots,n
\end{split}
\end{equation}

Ainsi, pour chaque $n>=1$, on a definit un march\'e \`a $n$ p\'eriodes de pas constant $\frac{T}{n}$.

\subsubsection{Proposition}
Le march\'e est sans arbitrage ssi
\begin{equation}
d_n<1+R_n<u_n avec 
\left\{\begin{array}{rcl}
u_n=e^{b_n+\sigma_n}\\
d_n=e^{b_n-\sigma_n}
\end{array}\right.
\end{equation}

Dans ce cas, il existe une unique mesure de probabilit\'e risque neutre $Q^n$ d\'efinie par:
\begin{equation}
Q^n(z_i=1)=q_n=\frac{1+R_n-d_n}{u_n-d_n}
\end{equation}
En plus, le march\'e est complet.
\subsection{Valorisation dans le mod\`ele de Cor-Ross-Robinstein}
On consid\`ere un actif conditionnel $H_n=g(S_n^1)$
\begin{equation}
P_0(H_n)=S_0^0E^{Q^n}[\frac{H_n}{S_n^0}]
\end{equation}
\begin{equation}
S_0^0=1, S_n^0=(1+R)S_{n-1}^0=\ldots=(1+R)^n=(e^{\Omega\frac{T}{n}})^n=e^{nT}
\end{equation}
\begin{equation}
P_0(H_n)=\frac{1}{(1+R_n)^n}E^{Q_n}[H_n]=e^{-\Omega T}E^{Q_n}[H_n]=e^{-\Omega T}E^{Q_n}[g(S_n^1)]
\end{equation}
Sous $Q^n$, $(\frac{z_{j+1}}{2})$ est une suite de V.A. de Bernouille de param\`etre $q_n=\frac{1+R_n-d_n}{u_n-d_N}$
\begin{equation}
Q^n(\sum_{i=1}{n}\frac{1+z_i}{2}=j)=C_n^j(q_n)^j(1-q_n)^{n-j},(0<=j<=n)
\end{equation}
\begin{equation}
P_0(H_n)=e^{-RT}\sum_{i=0}^n C_n^i(q_n)^i(1-q_n)^{n-i}g(i(u_n)^i(d_n)^{n-i})
\end{equation}
\begin{equation}
P_k(H_n)=e(-R(1-\frac{k}{n})T)\sum_{i=0}^{n-k}C_n^iq_n^i(1-q_n)^{n-k-i}g(S_k^iu_n^id_n^{n-k-i})
\end{equation}

\subsubsection{Proposition}

\begin{equation}
\begin{split}
P_k(H_n)&=e^{-R\frac{T}{n}}E^{Q_n}[P_{k+1}(H_n)|\mathcal{F}_k],k=0,\ldots,n-1\\
&=e^{-\frac{RT}{n}}[q_nP_{k+1}(H_n)_{u_n}+(1-q_n)P_{k+1}(H_n)_{d_n}]
\end{split}
\end{equation}
\begin{equation}
\begin{split}
E^{Q_n}[P_{k+1}(H_n)|\mathcal{F}_k]=E^{Q_n}[S_{k+1}^0E^{Q_n}[\frac{H_n}{S_n^0}|\mathcal{F}_{k+1}]|\mathcal{F}_k]\\
&=S^0_{k+1}E^{Q_n}[\frac{H_n}{S_n^0}|\mathcal{F}_k]\\
&=e^{R\frac{T}{n}}P_k[H_n]
\end{split}
\end{equation}

\textbf{Remarque}:
La valorisation et la couverture dans le cas multi-p\'eriode se r\'eduit \`a une s\'equence de valorisation/couverture dans le cas d'une seule p\'eriode.

En effet, $P_k(H_n)$ est le juste prix de l'actif conditionnel $P_{k+1}(H_n)$.
\begin{center}
\begin{tablular}
 $\frac{kT}{n} $ & $\frac{(k+1)T}{n}$\\
 Actif risqu\'e: $S_k^1$ & $->u_nS_k^1$\\
 & $->d_nS_k^n$
\end{tabular}
\end{center}

Actif conditionnel:
\begin{center}
\begin{tabular}
$\frac{kT}{n}$ & $\frac{(k-1)T}{n}$ \\ 
$P_k(H_n)$ & $->P_{k+1}(H_n)$\\
&$->P_{k+1}(H_n)_{d_n}$
\end{tabular}
\end{center}

La strat\'egie de couverture s'\'ecrit donc
\begin{equation}
\theta_{k+1}^1=\frac{P_{k+1}{(H_n)_{u_n}}-P_{k+1}(H_n)_{d_n}}{u_nS_k^i-d_nS_k^i}
\end{equation}
On la tracte $(P_k(H_n),\theta_{k+1}^1)$.

\subsubsection{Exercise}
On consid\`re le mod\`ele suivant:
\begin{equation}
S_0^0=1->S_1^0=1+R->S_2^0=(1+R)^2
\end{equation}
\begin{equation}
S_0^1=2
\begin{split}
->4\left\{\begin{array}\{rcl}
8\\
2
\end{array}\right.\\
->1
\left\{\begin{array}
2\\
\frac{1}{2}
\end{array}\right.
\end{split}
\end{equation}
1) Est-ce que le march\'e est viable/complet?
2) Calculer le prix d'un call sur l'actif risqu\'e $S^1$ de maturit\'e $2$ et de strike $1$.

\end{document}