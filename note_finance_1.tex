\documentclass{article}
\linespread{1.5}

\usepackage{amsmath}
\usepackage{amsfonts}
\usepackage{mathrsfs}

\title{Mineur Finance 1}

\begin{document}
\maketitle

\tableofcontents
\pagebreak

\begin{document}
\section{Cadre de valorisation des produite devive}
\subsection{Introduction}
\subsection{Hypothese}
\subsection{Propriete}
\subsection{Options americaines}
Option am\'ericaine similaire aux options europeens mais l'exercise du droit pourra se faire \`a toute date $t$ avant maturit\'e $T$.

On note $CallAmeric_t(T, K)$ et $PutAmeric_t(T, K)$ les prix des calls et puts americain au $t$.

On a 
\begin{equation}
CallAmer_t(T,K) >= Call_t(T, K)
\end{equation}

\begin{equation}
PutAmer_t(T, K) >= Put_t(T, K)
\end{equation}

Propri\'et\'e:
En l'absence de divident, on a:

\begin{equation}
CallAmer_t(T, K) = Call_t(T, K)
\end{equation}

Preuve:

\begin{equation}
Call_t(T, K) >= (S_t- K)_+
\end{equation}

Il n'est pas optimal d'exercer avant la maturit\'e. Donc $CallAmer_t(T, K) = Call_t(T, K)$

Remarque: Pour le Put, la propri\'et\'e n'est pas vraie.

\begin{equation}
Put_t(T, K) <= KB(t, T)
\end{equation}

Pour un certain scenario, le prix de put inferieur au prix d'exercise. $Put_t(T, K)<(S_t-K)_+$.

Pour $S_t$ tel que $KB_t(T)< K-S_t <=> S_t < (1-B_t(T))K$
Donc, on a
\begin{equation}
Put_t(T, K)< K-S_t
\end{equation} 

\section{Th\'eorie de la valorisation dans le cas discr\`et \`a une p\'eriode}












\end{document}